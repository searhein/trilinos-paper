\todo{Describe new linear solver packages, ShyLU, FROSch, new packages similar to old packages, new capabilities in these packages (two level DD), GPU supported features}

Trilinos have several linear solver capabilities that vary from direct solvers, iterative solvers, preconditioners that are local to a compute node to scalable domain decomposition and multigrid methods. Almost all the capabilities described here are focused on the current generation Trilinos with the Tpetra stack. All of these solver capabilities are built on top of Kokkos and are GPU capable to varying degrees. We present any exception to this on the detailed description below.

\subsection{Iterative Linear Solvers}


\subsection{Domain Decomposition or Schur Complement Type Methods: Ifpack2 and ShyLU}


\subsection{Multilevel Domain Decomposition: FROSch}


\subsection{Multigrid Methods: MueLu}

MueLu is a flexible and scalable high-performance multigrid solver library.
It provides a variety of multigrid algorithms for problems ranging from Poisson-like operators over elasticity, convection-diffusion, and Navier-Stokes to Maxwell’s equations.
Besides its strong focus on aggregation-based algebraic multigrid (AMG) methods,
MueLu comes with specialized capabilities for (semi-)structured grids to perform (semi-)coarsening along grid lines,
yet forming the coarse operator via a Galerkin product (in contrast to classical geometric multigrid).
MueLu is extensible and allows for the research and development of new multigrid preconditioning methods.
Its weak and strong scalability even for vector-valued PDEs on unstructured meshes
up to 131,000 cores of a Cray XC40 and one million cores of a Blue Gene/Q system~\cite{Lin2017a,Thomas2019a}. \todo{CG: Find some newer runs.}

\todo{Should we comment on the current state of Kokkos in MueLu? CG: Yes}

MueLu provides several approaches to constructing and solving the multilevel problem:

\begin{itemize}
\item \emph{Algebraic smoothed aggregation approach}~\cite{Vanek1996a}:
The matrix graph is colored to create aggregates (groups) of nodes.
These aggregates define a preliminary projection operator.
A final projection operator is created by applying a smoother to the preliminary operator.

\item \emph{Algebraic multigrid for Maxwell’s equations}:
\todo{CG: I can write something}

\item \emph{Semi-coarsening algebraic multigrid approach}~\todo{Add citation}:
Specialized aggregation procedures for three-dimensional meshes generated by extrusion of a two-dimensional unstructured grid
allow to first coarsen in the direction of extrusion to reduce the system to a two-dimensional representation and then perform classical aggregation-based AMG
for the remaining coarsenings.

\item \emph{AMG for (semi-)structured grids}:
Structured aggregation allows to identify coarse points with a user-given coarsening rate and compute interpolation operators.
The coarse operators are then formed via a Galerkin product to avoid remeshing on the coarse levels.
This work has been extended to semi-structured grids to leverage structured-grid computational performance~\cite{Mayr2022a}.

\end{itemize}

Several resources provide insight into MueLu:
An overview is given on the MueLu website\footnote{\url{https://trilinos.github.io/muelu.html}}.
The MueLu User's Guide~\cite{BergerVergiat2023a} summarizes installation instructions and a reference to most of MueLu's configuration parameters.
The MueLu Tutorial~\cite{Mayr2023b} introduces beginners and experts to various topics in MueLu and shows how to solve or precondition different linear systems using MueLu.
Details on the compatibility of MueLu and its predecessor ML~\cite{Heroux2005a,Gee2006a} can be found in the MueLu User's Guide~\cite{BergerVergiat2023a}.

Besides its C++ API, MueLu offers a MATLAB interface, MueMex, to provide access to MueLu's aggregation and solver routines from MATLAB
MueMex allows users to setup and solve arbitrarily many problems with either MueLu as a preconditioner, Belos as a solver and Epetra or Tpetra for data structures.



\subsection{Direct Linear Solvers}

\subsection{Native Direct Linear Solvers}

\subsection{Interfaces to Third-party Direct Linear Solvers}

\subsection{Eigensolvers}
\todo{Do we want that here or somewhere else?}

\todo{CG: How about Stratimikos or Stratimikos2?}