%%
%% This is file `sample-acmsmall.tex',
%% generated with the docstrip utility.
%%
%% The original source files were:
%%
%% samples.dtx  (with options: `acmsmall')
%% 
%% IMPORTANT NOTICE:
%% 
%% For the copyright see the source file.
%% 
%% Any modified versions of this file must be renamed
%% with new filenames distinct from sample-acmsmall.tex.
%% 
%% For distribution of the original source see the terms
%% for copying and modification in the file samples.dtx.
%% 
%% This generated file may be distributed as long as the
%% original source files, as listed above, are part of the
%% same distribution. (The sources need not necessarily be
%% in the same archive or directory.)
%%
%%
%% Commands for TeXCount
%TC:macro \cite [option:text,text]
%TC:macro \citep [option:text,text]
%TC:macro \citet [option:text,text]
%TC:envir table 0 1
%TC:envir table* 0 1
%TC:envir tabular [ignore] word
%TC:envir displaymath 0 word
%TC:envir math 0 word
%TC:envir comment 0 0
%%
%%
%% The first command in your LaTeX source must be the \documentclass
%% command.
%%
%% For submission and review of your manuscript please change the
%% command to \documentclass[manuscript, screen, review]{acmart}.
%%
%% When submitting camera ready or to TAPS, please change the command
%% to \documentclass[sigconf]{acmart} or whichever template is required
%% for your publication.
%%
%%
\documentclass[acmsmall]{acmart}

%%
%% \BibTeX command to typeset BibTeX logo in the docs
\AtBeginDocument{%
  \providecommand\BibTeX{{%
    Bib\TeX}}}

%% Rights management information.  This information is sent to you
%% when you complete the rights form.  These commands have SAMPLE
%% values in them; it is your responsibility as an author to replace
%% the commands and values with those provided to you when you
%% complete the rights form.
\setcopyright{acmcopyright}
\copyrightyear{2023}
\acmYear{2023}
\acmDOI{XXXXXXX.XXXXXXX}

\usepackage{color}


\newcommand{\todo}[1]{\textcolor{orange}{TODO: #1}}
\DeclareMathOperator*{\argmin}{\arg\!\min}


\usepackage{listings}
\lstset{language=C}


%%
%% These commands are for a JOURNAL article.
\acmJournal{JACM}
\acmVolume{37}
\acmNumber{4}
\acmArticle{111}
\acmMonth{8}

%%
%% Submission ID.
%% Use this when submitting an article to a sponsored event. You'll
%% receive a unique submission ID from the organizers
%% of the event, and this ID should be used as the parameter to this command.
%%\acmSubmissionID{123-A56-BU3}

%%
%% For managing citations, it is recommended to use bibliography
%% files in BibTeX format.
%%
%% You can then either use BibTeX with the ACM-Reference-Format style,
%% or BibLaTeX with the acmnumeric or acmauthoryear sytles, that include
%% support for advanced citation of software artefact from the
%% biblatex-software package, also separately available on CTAN.
%%
%% Look at the sample-*-biblatex.tex files for templates showcasing
%% the biblatex styles.
%%

%%
%% The majority of ACM publications use numbered citations and
%% references.  The command \citestyle{authoryear} switches to the
%% "author year" style.
%%


%%
%% end of the preamble, start of the body of the document source.
\begin{document}

%%
%% The "title" command has an optional parameter,
%% allowing the author to define a "short title" to be used in page headers.
\title{An Update to the Overview of the Trilinos Library}

%%
%% The "author" command and its associated commands are used to define
%% the authors and their affiliations.
%% Of note is the shared affiliation of the first two authors, and the
%% "authornote" and "authornotemark" commands
%% used to denote shared contribution to the research.
\author{Sivasankaran Rajamanickam}
\email{srajama@sandia.gov}
\orcid{1234-5678-9012}
\author{Michael Heroux}
\email{maherou@sandia.gov}
\author{Kim Liegeois}
\email{knliege@sandia.gov}
\orcid{0000-0002-1182-4078}
\author{Roger Pawlowski}
\email{rppawlo@sandia.gov}
\author{Eric Phipps}
\email{etphipp@sandia.gov}
\author{Mauro Perego}
\email{mperego@sandia.gov}
\author{Luc Berger-Vergiat}
\email{lberge@sandia.gov}
\orcid{0000-0001-5550-3527}
\author{Christopher M. Siefert}
\email{csiefer@sandia.gov}
\author{Christian Glusa}
\email{caglusa@sandia.gov}
\author{Jonathan Hu}
\email{jhu@sandia.gov}
\author{Graham Harper}
\email{gbharpe@sandia.gov}
\author{Paul Kuberry}
\email{pakuber@sandia.gov}
\orcid{0000-0002-2426-4591}
\author{Curtis C. Ober}
\email{ccober@sandia.gov}
\author{Heidi K. Thornquist}
\email{hkthorn@sandia.gov}
\author{James M. Willenbring}
\email{jmwille@sandia.gov}
\orcid{0000-0002-0418-9264}
\author{Roscoe A. Bartlett}
\email{rabartl@sandia.gov}
\orcid{0000-0002-3831-8060}
\author{Denis Ridzal}
\email{dridzal@sandia.gov}
\author{Drew P. Kouri}
\email{dpkouri@sandia.gov}
\author{Ichitaro Yamazaki}
\email{iyamaza@sandia.gov}
\author{Nathan V. Roberts}
\email{nvrober@sandia.gov}
\orcid{0000-0003-1536-0749}

\author{New Sandia Author}
\email{TBD@sandia.gov}

\affiliation{%
  \institution{Sandia National Laboratories}
  \streetaddress{1515, Eubank Blvd SE}
  \city{Albuquerque}
  \state{New Mexico}
  \country{USA}
  \postcode{87123}
}

\author{Alexander Heinlein}
\affiliation{%
  \institution{Delft University of Technology}
  \streetaddress{Mekelweg 4}
  \postcode{2628CD}
  \city{Delft}
  \country{Netherlands}
}
\email{a.heinlein@tudelft.nl}
\orcid{0000-0003-1578-8104}

\author{Matthias Mayr}
\affiliation{%
  \institution{University of the Bundeswehr Munich}
  \streetaddress{Werner-Heisenberg-Weg 39}
  \city{85577 Neubiberg}
  \country{Germany}}
\email{matthias.mayr@unibw.de}
\orcid{0000-0002-2780-1233}

\author{New Non Sandia Author}
\affiliation{%
  \institution{TBD}
  \city{TBD}
  \country{TBD}
}





%%
%% By default, the full list of authors will be used in the page
%% headers. Often, this list is too long, and will overlap
%% other information printed in the page headers. This command allows
%% the author to define a more concise list
%% of authors' names for this purpose.
\renewcommand{\shortauthors}{Developer et al.}

%%
%% The abstract is a short summary of the work to be presented in the
%% article.
\begin{abstract}
 This is an update to the ``Overview of Trilinos document.'' Trilinos framework has undergone substantial changes to support new applications and new hardware architectures. We describe the new organization, features and design of Trilinos here.
\end{abstract}

%%
%% The code below is generated by the tool at http://dl.acm.org/ccs.cfm.
%% Please copy and paste the code instead of the example below.
%%
%\begin{CCSXML}
%<ccs2012>
% <concept>
%  <concept_id>10010520.10010553.10010562</concept_id>
%  <concept_desc>Computer systems organization~Embedded systems</concept_desc>
%  <concept_significance>500</concept_significance>
% </concept>
% <concept>
%  <concept_id>10010520.10010575.10010755</concept_id>
%  <concept_desc>Computer systems organization~Redundancy</concept_desc>
%  <concept_significance>300</concept_significance>
% </concept>
% <concept>
%  <concept_id>10010520.10010553.10010554</concept_id>
%  <concept_desc>Computer systems organization~Robotics</concept_desc>
%  <concept_significance>100</concept_significance>
% </concept>
% <concept>
%  <concept_id>10003033.10003083.10003095</concept_id>
%  <concept_desc>Networks~Network reliability</concept_desc>
%  <concept_significance>100</concept_significance>
% </concept>
%</ccs2012>
%\end{CCSXML}

%\ccsdesc[500]{Computer systems organization~Embedded systems}
%\ccsdesc[300]{Computer systems organization~Redundancy}
%\ccsdesc{Computer systems organization~Robotics}
%\ccsdesc[100]{Networks~Network reliability}

%%
%% Keywords. The author(s) should pick words that accurately describe
%% the work being presented. Separate the keywords with commas.
\keywords{Scientific Software Frameworks}

\received{15 April 2023}
%\received[revised]{12 March 2009}
%\received[accepted]{5 June 2009}

%%
%% This command processes the author and affiliation and title
%% information and builds the first part of the formatted document.
\maketitle

\section{Introduction}

%\todo{Add introduction here. Add a subsection on Trilinos organization as a set of five products.}

Trilinos is a community-developed, open source software framework that facilitates building large-scale, complex, multiscale, multiphysics engineering and scientific problems. While Trilinos can run on small workstations to large supercomputers, the typical use of Trilinos is on the leadership class systems with new or emerging hardware architectures.

% History
Trilinos was originally conceived as framework of three packages for distributed memory systems. The original Trilinos publication~\cite{Heroux2005a} describes the motivation and the philosophy behind Trilinos and the capabilities that existed in Trilinos at that time. The second Trilinos overview publication~\cite{Heroux2012} introduced the expanded set of capabilities then included in Trilinos as well as the Trilinos strategic goals. Trilinos today is similar to the Trilinos that was envisioned two decades ago in some aspects. However, Trilinos today is also very different in several other aspects. These changes were necessitated by the changes in programming models, application needs, hardware architectures, and algorithms. Trilinos has grown from a library of three packages to a library with more than fifty packages with functionality and features supporting a wide range of applications.

% Purpose
This article is an attempt to capture a snapshot of where Trilinos is today as opposed to eighteen and eleven years ago when the original Trilinos articles were written. We will focus on the major developments within Trilinos in the last decade, new features and functionality that has been added to enable scientific and engineering applications. This article will be an overview of the features and we refer to the extensive reference list for the details of these features. We are also cognizant of the fact that as a software that is actively developed this article could become outdated even before its publication. Hence, we will focus on the high level features and project that we expect to remain stable for several years.

%Product and package structure
The functionalities in Trilinos are organized in two levels. The first one is \textit{package}. A package in Trilinos has a well-defined set of unique capabilities that is important for a scientific or an engineering application. Packages also have a set of expectations such as having a responsible point of contact or a package lead, software engineering expectations such as documentation, continuous integration testing, clearly defined dependencies, using the Trilinos infrastructure for building and installation etc. Recently, we have aggregated the fifty or more packages into five \textit{product areas} for organizational ease. The five product areas are data services, discretizations, linear solvers, embedded nonlinear analysis and tools, and framework. These product areas are collection of packages that share a common objective (e.g., solving a linear system), a sub-community within Trilinos, and in some cases common interfaces. We briefly describe these areas here.

\paragraph{Data Services} The data services product area covers all aspects of creating, distributing or mapping data to processing elements (cores, threads, nodes), load balancing, and redistributing data. Data services also includes Trilinos abstractions for data such as the Petra object model, and its concrete implementation called Tpetra. On a modern accelerator-based compute node the abstractions provided by the Kokkos library becomes critical for Tpetra. Section \ref{sec:data_services} describes these features in detail.
 
\paragraph{Linear Solvers} The wide variety of applications that use Trilinos need a diverse set of linear solvers. Trilinos has support for both iterative and direct linear solvers. There are a number of preconditioner options from multithreaded or performance portable node-level preconditioners to scalable multilevel domain decomposition or multigrid preconditioners. The preconditioners and solvers use the data abstractions from the data services product area. Section \ref{sec:lin_solve} describes these features in detail.

\paragraph{Discretization and Analysis Tools} This product provides tools for the discretization of differential equations, as well as algorithms to perform analysis. In particular, the product supports mesh-free and mesh-based spatial discretizations, time integration, solution of nonlinear equations, parameter continuation, bifurcation tracking, numerical optimization and uncertainty quantification. The product includes cross-cutting utilities for algorithmic differentiation and for managing directed graphs of evaluation kernels. Section \ref{sec:discretization} describes these features in detail.

%\paragraph{Nonlinear Analysis} The nonlinear analysis product area provides high level algorithms for computational simulation and design. Capabilities include solvers for nonlinear equations, time integration, parameter continuation, bifurcation tracking, optimization and uncertainty quantification. This capability area also provides lower level utility packages to evaluate quantities of interest required by the analysis algorithms. Capabilities include automatic differentiation technology to evaluate derivatives and embedded ensemble propagation for uncertainty quantification. Section \ref{sec:nonlin_solve} describes these features in detail.

\paragraph{Framework} The Framework Product is different than the other Trilinos Products in that most of the resources and services are not associated with Trilinos packages. The Framework Product rather is focused primarily on activites such as developing and maintaining infrastructure for automated testing and documentation, as well as associated workflows. A small number of infrastructure and cross-cutting packages are also associated with the Framework, including Teuchos and PyTrilinos. Section \ref{sec:framework} describes these features in detail.

%Article organization
This article describes Trilinos' product areas and their packages with a focus towards providing an overview of recent developments. We also briefly touch upon the Trilinos community (Section \ref{sec:community}) and software engineering issues with respect to Trilinos.



\section{Data Services}
\label{sec:data_services}
% MMW: I'd order these subsections roughly in order of the software stack... most fundamental to most complex

\todo{Describe performance portability... I think this should go into the intro section.}

\todo{@Siva: do this!}

\subsection{Kokkos Kernels}\label{subsec:kk}
Kokkos Kernels~\cite{rajamanickam2021kokkoskernels} is part of the Kokkos ecosystem
~\cite{trott2021kokkos} and provides abstractions and implementations of node local mathematical 
kernels widely used across packages in Trilinos. As a member of the Kokkos
ecosystem, Kokkos Kernels is tightly integrated with Kokkos features and aims at
delivering performance portable algorithms across major CPU and GPU based HPC systems.
Kokkos Kernels serves as a major point of integration for vendor optimized libraries
such as cuBLAS, cuSPARSE, rocBLAS, rocSPARSE, MKL, ARMpl and others.

The implementation of Kokkos Kernels algorithms leverages the hierarchical parallelism
exposed by the Kokkos library~\cite{kim2017designing} and increasingly provides coverage
for stream callable kernels. To ensure flexibility for the distributed libraries that
call its algorithms, Kokkos Kernels provides thread safe and asynchronous
implementations for most of its kernels. Due to its node local nature, Kokkos Kernels does 
not rely on MPI or other communication libraries unlike numerous other packages in Trilinos.

The capabilities that Kokkos Kernels provides can be divided in four major categories:
1. BLAS algorithms, 2. sparse linear algebra and preconditioners, 3. graph algorithms, and
4. batched dense and sparse linear algebra~\cite{liegeois2023performance}. The main
points of integration of Kokkos Kernels in Trilinos are Tpetra for the dense and sparse
linear algebra capabilities, Ifpack2 for the preconditioners and batched algorithms,
the multigrid package MueLu that relies on these features both directly and indirectly
as well as on some specialized algorithms such as graph
coloring/coarsening~\cite{kelley2022parallel} and fused Jacobi-SpGEMM (generalized sparse matrix matrix multiply) kernels.

Similarly to the Kokkos library, Kokkos Kernels is developed in its own GitHub
repository\footnote{https://github.com/kokkos/kokkos-kernels} outside of the Trilinos
GitHub repository. Every version of the library is integrated and tested in Trilinos
as part of the Kokkos ecosystem release process. Additional information on Kokkos
Kernels capabilities can be found here~\cite{deveci2018multithreaded,wolf2017fast}.

\todo{I'm not sure if we want to include Teuchos, but it might be necessary since it is omnipresent in the use of Trilinos.  We probably should expand this a bit more.}

% MMW I don't think Teuchos should be under framework.  I think we want to stick to logical groupings versus load-balanced ones.
\subsection{Teuchos}

Teuchos provides a suite of common tools for many Trilinos packages.
These tools include memory management classes~\cite{bartlett2010} such as
\todo{Discuss the relevancy of bringing smart pointer in 2024, while requiring c++17...}
``smart'' pointers and arrays,
\todo{What about the MPI wrapper ? For instance, we quite often use the Teuchos serial comm for "single rank" unit tests.}
``parameter lists'' for communicating hierarchical lists of parameters between library or application layers,
templated wrappers for the BLAS and LAPACK, XML parsers, and other utilities.
They provide a unified ``look and feel'' across Trilinos packages, and help avoid common programming mistakes.

\subsection{Tpetra}\label{subsec:tpetra}
Tpetra \cite{hoemmen2015tpetra} provides the distributed-memory
infrastructure for sparse linear algebra computations.  It implements
distributed-memory linear algebra objects, such as sparse graphs,
sparse matrices and dense vectors, using Kokkos for local data
storage.  Distributed-memory sparse linear algebra operations, such as
a sparse matrix-vector product, are implemented through on-node calls
to Kokkos Kernels and inter-node MPI communication.   Tpetra features
include:
\begin{itemize}
\item \textit{Maps} --- distributions of objects over MPI ranks.
\item \textit{(Multi)Vectors} --- storage of dense vectors or collections of
vectors (multivectors) and associated BLAS-1 like kernels (e.g., dot
products, norms, scaling, vector addition, pointwise vector
multiplication) as well as tall skinny QR (TSQR)  factorization for multivectors.
\item \textit{Import/export} --- moving vector, graph and matrix data
between different distributions (maps).  This is key for performing
halo/boundary exchanges, as well as other kernels such as
sparse matrix-matrix multiplication.
\item \textit{Sparse graphs} --- in compressed sparse row (CSR)
format.  Graphs also include import/export objects for use in
halo/boundary exchanges associate with the graph.
\item \textit{Sparse matrices} --- in compressed sparse row (CSR) and
block compressed sparse row (BSR) formats.  Associated kernels include
sparse matrix vector product (SPMV), sparse matrix-matrix
multiplication (SPGEMM) and triple-product, sparse matrix-matrix addition, sparse matrix transpose, diagonal extraction,
Frobenius norm calculation, and row/column scaling.
\end{itemize}

\todo{@Brian/Chris: do you want to expand this a bit?}

\subsection{Thyra}
The Thyra library provides abstractions for use in writing high level abstract numerical algorithms (ANAs). Thyra is used by other Trilinos packages for implementing ANAs such as solvers for linear equations, nonlinear equations, bifurcation and stability analysis, optimization and uncertainty quantification. The abstractions hide concrete linear algebra and solver implementations so that the algorithms can be written once and reused across many application codes. For example, one could write a Krylov solver and it could be used with multiple concrete linear algebra implementations such as Tpetra in Trilinos, Epetra in Trilinos or the vector/matrix objects in PETSc.

Thyra is organized into three sets of abstractions depending on the type of operation required. All are templated on the scalar type. The sets are:
\begin{itemize}
\item \emph{Vector/Operator Interfaces:} The first level is an abstraction for vectors and operators \cite{Bartlett2007}. This set includes abstractions for vector spaces, vectors, multivectors and linear operators. Thyra provides a default set of vector/vector and vector-operator operations for standard linear algebra tasks such ad AXPY from BLAS. Defining an extensible interface such that users can add arbitrary vector/matrix operations and maintain performance can be difficult. A general tool to achieve this is the Trilinos utility package called RTOp \cite{rtop}. RTOp provides an interface for users to add new Thyra operations. RTOp accepts a vector of Thyra input objects and a vector of Thyra output objects along with a functor. The RTOp will apply the functor to each element of the data.

\item \emph{Operator Solve Interfaces:} The second set of abstractions supply abstract linear solvers that can be attached to operators. The interfaces include base classes for preconditioners, preconditioner factories, linear operators with a solver, and factories for linear operators with a solver. The factories build and update the preconditioners and linear solvers as needed. The solver interfaces allow one to wrap any concrete linear solver or preconditioner for use with the ANAs. Trilinos provides a separate library named Stratimikos that wraps all of the Trilinos preconditioners and linear solvers in Thyra abstractions. This library is driven by the Teuchos::ParameterList options.

\item \emph{Nonlinear Interfaces:} The final set of abstractions support nonlinear operations. Thyra provides an interface that wraps nonlinear solvers. Implicit time integration and optimization solvers typically require nonlinear solves and can abstract away the concrete implementation via Thyra. The NOX package in Trilinos provides a concrete implementation of this interface. The final interface in Thyra is for querying a physics application for common objects required by high level Newton-based solvers. The Thyra::ModelEvaluator interface allows codes to ask the application to evaluate residuals, Jacobians, parametric sensitivities, Hessians and post processed quantities/derivatives of interest. The interface is designed to be stateless (i.e. consecutive calls with the same inputs should produce the same outputs). This design was chosen to avoid pitfalls in complex application codes interacting with general nonlinear algorithms. The design does not preclude its use in stateful applications that have a ``memory'' between time steps, but rather forces the explicit use of the stateful information in the interface.
\end{itemize}





\section{Discretization}
\label{sec:discretization}
The discretization and analysis tools product contains several packages to handle discretizations of differential equations using high-order finite elements, finite volumes or generalized moving least squares for spatial discretization, as well as many explicit and implicit schemes for time integration. The product also include tools for “analysis beyond simulation”, which aims to automate many computational tasks that are often performed by application code users by trial-and-error or repeated simulation. Tasks that can be automated include performing parameter studies, sensitivity analysis, time step size control, and locating instabilities. The product also includes tools for numerical optimization and uncertainty quantification. Many of these capabilities rely on utility packages including algorithmic differentiation, tical to the analysis algorithms and the abstraction layers and interfaces for application callbacks.


\subsection{Intrepid2}
Intrepid2 provides interoperable tools for compatible discretizations of PDEs; it is a performance-portable re-implementation and extension of the legacy Intrepid package \cite{bochev2012}. Intrepid2 mainly focuses on local assembly of continuous and discontinuous finite elements. It also provides limited capabilities for finite volume discretization.  Intrepid2 works on batches of elements (cells), and provides tools to efficiently compute discretized linear functionals (e.g., right-hand-side vectors) and differential operators (e.g., stiffness matrices) at the element level. Intrepid2 implements compatible finite element spaces of various polynomial orders for $H({\rm grad})$, $H({\rm curl})$, $H({\rm div})$ and $L^2$ function spaces on triangles, quadrilaterals, tetrahedrons, hexahedrons, wedges and pyramids. It provides both Lagrangian basis functions and hierarchical basis functions \cite{fuentes2015} and it implements performance optimizations (e.g., sum factorizations) exploiting the underlying structure of the problem (e.g., tensor-product elements or other symmetries).  The degrees of freedom of $H({\rm div})$ and $H({\rm curl})$ finite elements, as well as high-order $H({\rm grad})$ finite elements, depend on the global orientation of edges and faces and Intrepid2 provides orientation tools for matching the degrees of freedom on shared edges and faces. It also provides interpolation-based projection tools for projecting functions in $H({\rm grad})$, $H({\rm curl})$, $H({\rm div})$ and $L^2$ to the respective discrete spaces. Intrepid2 implements these capabilities through these classes:
\begin{itemize}
\item \emph{CellTools:} This class provides geometric operations on the reference and physical frame. Includes computation of tangents and normals to edges/faces in the physical frame, computation of Jacobian of the reference-to-physical frame maps, and other metric computations. 
\item \emph{CubatureFactory:} This class provides several quadrature rules (called \emph{cubatures} in Intrepid2) of various degrees of accuracy for approximating integrals over the elements and their boundaries.
\item \emph{Basis:} This is the base class for a variety of basis functions for compatible finite element spaces. Each class includes a \texttt{getValues()} method that computes the value of the basis functions or their derivatives (e.g., gradient for $H({\rm grad})$ functions, curl for $H({\rm curl})$ functions) at a set of input points. The implementation of \texttt{getValues()} can be very different depending on the basis. Specific optimizations are available for tensor-product elements.  Additionally, there is a \texttt{BasisFamily} class with a convenience method, \texttt{getBasis()}, which constructs a basis depending on a template argument specifying the type of basis (hierarchical or nodal, e.g.), the cell topology and function space on which it is defined, and its polynomial degree.
\item \emph{OrientationTools:} This class provides methods to orient the basis functions based on the global orientation of edges and faces, determined by the global numbering of the cell vertices. This is achieved by building a linear operator (a permutation for tensor-product elements) that encodes the orientation of a particular cell, and applying that operator to the reference basis functions.
\item \emph{ProjectionTools:} This class provides methods for interpolation-based projections of a given function into a compatible finite element space or between compatible finite element spaces \cite{demkowicz2007}.  The provided projections commute with the corresponding differential operators if the quadrature rules can exactly integrate the functions being projected. As an example, projecting an $H({\rm grad})$ function into the $H({\rm grad})$ finite-element space and then taking its gradient gives the same result as taking the gradient of the function first, and then projecting the gradient into the $H({\rm curl})$ finite-element space.
\item \emph{FunctionSpaceTools:} This class provides transformations of fields from reference to physical frame and back, computation of measures on edges, faces and cells, scalar/vector/tensor multiplications and contractions for computing integrals.
\item \emph{IntegrationTools:} This class provides integration methods that can take advantage of tensor product structures in basis values, providing mechanisms for performance-portable, \emph{sum-factorized} assembly across $H({\rm grad})$, $H({\rm curl})$, $H({\rm div})$ and $L^2$ function spaces.  In the future, we plan to provide similar interfaces to support matrix-free discretizations.
\end{itemize}
Intrepid2 makes use of Kokkos containers to enable memory layouts that are adapted to the computational platform. Intrepid2 also uses Kokkos for its core computational kernels, enabling threaded execution across a variety of architectures. The data types used by Intrepid2 are templated; it is therefore possible to propagate Sacado types through Intrepid2 to perform automatic differentiation. Current development of Intrepid2 focuses on providing efficient matrix-free discretizations to enhance efficiency on GPU architectures. 

\subsection{Phalanx}
Phalanx is a local field evaluation library designed for equation assembly in partial differential equation (PDE) applications. The goal of Phalanx is to decompose a complex problem into a number of simpler problems with managed dependencies to support rapid development and extensibility of PDE codes \cite{Notz2012,pawlowski2012automating,pawlowski2012automatingpart2}. The data structures use Kokkos \cite{trott2021kokkos} for performance portability. Through the use of template metaprogramming, Phalanx supports arbitrary user defined data types and evaluation types. This feature allows for simple integration of automatic differentiation tools via operator overloaded scalar types from the Sacado library \cite{phipps2022automatic}. From a simple equation definition, quantities such as Jacobians, Jacobian-vector products, Hessians and parameter sensitivities can be evalauted to machine precision. These quantities can be used by other Trilinos packages for operations including Newton-based nonlinear solves, gradient-based optimization, constraint enforcement and bifurcation analysis \cite{pawlowski2012automating,pawlowski2012automatingpart2}.

Phalanx uses a graph-based design to manage data dependencies. The runtime defined directed acyclic graph allows for rapid prototyping in a production environment where simple interfaces for analysts, flexible models/data structures and integration of non-trivial Third-party libraries are paramount. Phalanx is used in a number of large scale parallel codes including Albany \cite{Salinger2016}, Charon \cite{CharonUsersManual2020} Drekar \cite{Crockatt2022,Miller2019,Shadid2016mhd} and EMPIRE \cite{Bettencourt2021}.

In recent years, Phalanx has been extended to provide utilities for performance portability under automatic differentiation. For example, Phalanx provides tools for building and managing a Kokkos view-of-views on device without the use of unified virtual memory and provides utilities for running virtual functions on device. In the future, these utilities may be split out into a separate package.

\subsection{Panzer}
\todo{Roger, please edit/expand}
The package provides global tools for finite element analysis. It handles continuous and discontinuous high-order compatible finite elements, as implemented in Intrepid2 on unstructured meshes. Panzer relies on Phalanx to manage with efficiency and flexibility the assembly of complex problems. Panzer also enables the solution of nonlinear problems, by interfacing with several Trlinos linear and nonlinear solvers. It computes derivatives and sensitivities through automatic differentiation (Sacado). It supports both Epetra and Tpetra data structures and achieves performance portability through the Kokkos programming model.

\subsection{Compadre}
The Compadre package provides tools for the approximation of linear operators (including point evaluation and derivatives), given the location of samples of a function over an unstructured cloud of points. The resulting stencils, when applied directly to samples of the function at these locations, provides an approximation of the linear operator acting on the function at the point(s) queried. This is useful for meshed and meshless data transfer applications. Values of the function at the specified locations can also be viewed as unknowns, in which case the solution returned by Compadre can be used as a stencil for meshless discretization of PDEs. 

The package uses generalized moving least squares (GMLS) for approximating functionals. We plan on implementing other meshless methods like radial basis functions in the future.  A brief description of GMLS follows, but technical details can be found in \cite{mirzaei2012generalized,wendland2004scattered}.

Consider $\phi$ of function class $\mathbf{V}$. Consider a collection of samples $\Lambda = \left\{\lambda_i(\phi)\right\}_{i=1}^{N}$ corresponding to a quasi-uniform\cite{wendland2004scattered} collection of data sites $\mathbf{X}_h = \left\{\mathbf{x}_i\right\} \subset \mathbb{R}^d$ characterized by fill distance $h$. To approximate a given linear target functional $\tau_{\tilde{x}}$ associated with a target site $\tilde{x}$, we seek a reconstruction $p \in \mathbf{V}_h$, where $\mathbf{V}_h \subset \mathbf{V}$ is a finite dimensional space chosen to provide good approximation properties, with basis $\mathbf{P}=\left\{P\right\}_{i=1}^{dim(V_h)}$. We perform this reconstruction in the following weighted $\ell_2$ sense:

\begin{equation}
\label{gmls}
p = \underset{{q \in \mathbf{V}_h}}\argmin \sum_{i=1}^N \left( \lambda_i(\phi) -\lambda_i(q) \right)^2 \omega(\lambda_i,\tau_{\tilde{x}})
\end{equation}
where $\omega$ is a locally supported positive function. Compadre offers several choices for the weighting kernel $\omega = \Phi(|\tilde{x}-\mathbf{x}_i|)$, where $|\cdot|$ denotes the Euclidean norm.

With the optimal reconstruction $p$ computed, the target functional is approximated via $\tau_{\tilde{x}} (\phi) \approx \tau^h_{\tilde{x}} (\phi) := \tau_{\tilde{x}} (p)$. As an unconstrained $\ell_2$-optimization problem, this process admits the explicit form


\begin{equation}
\label{discreteTarget}
\tau^h_{\tilde{x}}(\phi) = \tau_{\tilde{x}}(\mathbf{P})^\intercal \left(\Lambda(\mathbf{P})^\intercal \mathbf{W} \Lambda(\mathbf{P})\right)^{-1} \Lambda(\mathbf{P})^\intercal \mathbf{W} \Lambda(\phi),
\end{equation}
where we denote:
\begin{itemize}
  \item $\tau_{\tilde{x}}(\mathbf{P}) \in \mathbb{R}^{dim(V_h)}$ is a vector with components consisting of the target functional applied to each basis function.
  \item $\mathbf{W} \in \mathbb{R}^{N \times N}$ is a diagonal matrix with diagonal entries consisting of $\left\{\omega(\lambda_i,\tau_{\tilde{x}})\right\}_{i=1,...,N}$.
  \item $\Lambda(\mathbf{P}) \in \mathbb{R}^{N \times dim(V_h)}$ is a rectangular matrix whose $(i,j)$ entry corresponds to the application of the $i^{th}$ sampling functional applied to the $j^{th}$ basis function.
  \item $\Lambda(\phi) \in \mathbb{R}^N$ is a vector consisting of the $N$ samples of the function $\phi$.
\end{itemize}
We note that by taking the contraction of the tensors appearing in \eqref{discreteTarget} and exploiting the compact support of $\omega$, we may interpret the output of the GMLS process as a finite difference-like stencil of the form 
\begin{equation}
\tau^h_{\tilde{x}}(\phi) = \sum_{\mathbf{x}_i \in B^\epsilon(\tilde{x})} \alpha_i \lambda_i(\phi),
\end{equation}
where $B^\epsilon(\tilde{x})$ denotes the $\epsilon$-ball neighborhood of the target site $\tilde{x}$. Therefore, GMLS admits an interpretation as an automated process for generating generalized finite difference methods on unstructured point clouds. The computational cost of solving the GMLS problem amounts to inverting a small linear system which may be assembled using only information from neighbors within the support of $\omega$, and construction of such stencils across the entire domain is embarrassingly parallel. 

Compadre allows users to control the weighting kernel ($\omega$), degree of the reconstruction basis ($\mathbf{V}_h$), the sampling functionals ($ \left\{\lambda_i(\phi)\right\}_{i=1}^{N}$), and the target operator ($\tau_{\tilde{x}}$); this allows control over smoothness of the reconstruction, locality of the resulting stencil, order of accuracy of the reconstruction (assuming regularity of the function embedded in the point cloud data), choice of what the sampled data or degrees of freedom represent, and linear operator action, respectively.

Selecting point evaluations for sampling functionals and target operator provides a traditional moving least squares reconstruction. As an example of a more exotic choice, it is possible to use an average vector normal integral over edges as the sampling functionals and a cell average integral as the target operator, enabling recovery of functions embedded in a Raviart-Thomas type representation and transferring them to a basis consistent with a finite-volume scheme.

While Compadre supports full space reconstruction in 1-3D, there is also additional support for select sampling functionals and target operators on 1D smooth manifolds embedded in 2D or 2D smooth manifolds embedded in 3D. Reconstruction on a manifold is done through an on-the-fly PCA calculation to determine principal directions tangent to the manifold. The curvature of the manifold is calculated through a reconstruction of the tangent and normal components to the calculated tangent plane, and the final function reconstruction is performed in the local chart. Utilities in the package handle mappings between local computed charts and the ambient higher-dimensional space.

Compadre's stencil generation involves independent problems to be solved in parallel at the team level with loops over the thread and vector level within each problem. This hierarchical parallelism is achieved with performance portability by using the Kokkos programming model and leveraging the batched QR with pivoting algorithm implemented in Kokkos Kernels.


\section{Linear Solvers}
\label{sec:lin_solve}
\input{linear_solvers}

\section{Nonlinear Analysis}
\label{sec:nonlin_solve}

The Nonlinear Analysis product area provides the top level algorithms for a computational simulation or design study.
These include nonlinear solvers, time integration, bifurcation tracking, stability analysis, parameter continuation, optimization, and uncertainty quantification.
A common theme of this collection is the philosophy of ``analysis beyond simulation'', which aims to automate many computational tasks that are often performed by application code users by trial-and-error or repeated simulation.
Tasks that can be automated include performing parameter studies, sensitivity analysis, calibration, optimization, time step size control, and locating instabilities.
This capability area additionally includes utilities for the nonlinear analysis. These include automatic differentiation tools that can provide the derivatives critical to the analysis algorithms and the abstraction layers and interfaces for application callbacks.

\subsection{Thyra}
\todo{Bartlett}

\subsection{NOX}
NOX, short for Nonlinear Object-oriented Solutions, provides robust and efficient algorithms for solving sets of nonlinear equations.
NOX implements a number of Newton-based globalization techniques including line search \cite{Pawlowski2006}, trust region \cite{Pawlowski2006,Pawlowski2008} and homotopy algorithms \cite{Coffey2003}.
Additionally, it provides lower and higher order models including Broyden, Anderson acceleration \cite{Walker2011} and tensor methods \cite{Bader2005}.
The algorithms have been designed for large scale parallel inexact linear solvers using Krylov methods and supports Jacobian-free Newton-Krylov variants.
The library interacts with application codes through the Thyra model evaluator interface.
NOX provides linear algrebra abstractions for applications to use custom implementations with the algorithms.
The library additionally contains abstractions for solvers, directions and line searches that allow users to further customize the algorithms.
NOX provides various stopping criteria including absolute, relative and weighted root mean square norms, as well as stagnation and NaN detection.
Applications can build custom stopping criteria within a tree-based logical structure and provide addtional criteria through the StatusTest abstraction.

\subsection{LOCA}
LOCA~\cite{Salinger2005}, short for the Library of Continuation Algorithms, provides techniques for computing families of solutions to nonlinear equations as well as methods for investigating their stability when these nonlinear equations define equilibria of dynamical systems.  It builds on the NOX nonlinear solver package to track solutions to sets of nonlinear equations as a function of one or more parameters (continuation).  Given an interface to NOX defining the nonlinear equations, all users must additionally provide is an ability to set the parameter values that are being varied.  LOCA provides several continuation methods, including pseudo-arclength continuation which allows tracking solution curves around turning points/folds.  Furthermore, LOCA has hooks to call the Anasazi eigensolver package to estimate leading eigenvalues of the linearization at each point along the continuation curve for linear stability analysis, including various spectral transformations highlighting eigenvalues in different regimes (e.g., largest magnitude, largest real, ...).  Finally, LOCA implements equations augmenting the original nonlinear equations to locate and track bifurcation points where linear stability changes (e.g., turning point, pitchfork, and Hopf bifurcations) as a function of additional parameters. Examples of using LOCA for stability and bifurcation analysis include stability of a differentially heated cavity \cite{Salinger2002}, design of chemical vapor deposition reactors \cite{Pawlowski2001}, flow instabilities in counterflowing jets \cite{pawlowski_salinger_shadid_mountziaris_2006} and ignition in laminar diffusion flames \cite{Shadid20061846}.

\subsection{Tempus}

Tempus (Latin meaning time as in “tempus fugit” -> “time flies”)
is the Trilinos time-integration package for advanced transient
analysis.  It includes various time integrators and embedded
sensitivity analysis for next-generation code architectures.  Tempus
provides “out-of-the-box” time-integration capabilities, which
allows users to quickly and easily incorporate time-integration
capabilities to their applications and switch between various time
integrators depending on the simulation needs.  Additionally, Tempus
provides “build-your-own” capabilities, which allows applications
to incorporate various Tempus components to augment or replace
application transient capabilities. Other capabilities include
embedded error analysis, sensitivity analysis, transient optimization
with ROL.

Tempus provides a general infrastructure for the time evolution of
solutions through a variety of general integration schemes.  Tempus
provides time integrators for explicit and implicit methods and for
first- and second-order ODEs.  It can be used from small systems of
equations (e.g., single ODEs for the time evolution of plasticity
models, and multiple ODEs for coupled chemical reactions) to
large-scale transient simulations requiring exascale computing
(e.g., flow fields around reentry vehicles and magneto-hydrodynamics).

Tempus has several components that can be used in concert or
individually, depending on the needs of the application.
\begin{itemize}
  \item Integrators are the time-loop structure for time integration
  and provide several features, e.g., control the advancement of
  the solution, selection of the next timestep size and solution
  output.

  \item Time Steppers are individual methods that advance the
  solution from one step to the next.  A variety of time steppers
  are available:
  \begin{itemize}
    \item Classic one-step methods (e.g., Forward Euler and Trapezoidal
    Method)
    \item Explicit Runge-Kutta methods (e.g., RK Explicit 4 Stage)
    \item Diagonally Implicit Runge-Kutta (DIRK) Methods (e.g.,
    general tableau DIRK and many specific DIRK/SDIRK methods)
    \item Implicit-Explicit (IMEX) Runge-Kutta Methods (e.g., IMEX
    RK SSP2, IMEX RK SSP3, and general tableau IMEX RK methods)
    \item Multi-Step Methods (i.e., BDF2)
    \item Second-order ODE Methods (e.g., Leapfrog, Newmark methods
    and HHT-Alpha)
    \item Steppers with subSteppers (e.g., operator-split and
    subcycling methods)
  \end{itemize}

  \item Solution History is used to maintain the solution during
  time-step failure, solution restart/output, interpolation of
  solution between time steps, and to provide the solution for
  transient adjoint sensitivities.

  \item Timestep Control and Strategies provide methods to select
  the time-step size based on user input and/or temporal error
  control (e.g., bounding min/max time-step size, relative/absolute
  maximum error, and timestep adjustments for output and checkpointing)

\end{itemize}
Additionally, Tempus has several mechanisms which allow users to
insert application-specific algorithms into Tempus components (e.g.,
through observers and creating derived classes).

\subsection{Piro}

Piro~\cite{osti_1231283} is the top-level, unifying package of the embedded nonlinear analysis capability area. 
The main purpose of the package is to provide driver classes for the common uses of Trilinos nonlinear analysis tools. 
These drivers all can be constructed similarly, with a \lstinline{Thyra::ModelEvaluator} and a \lstinline{Teuchos::ParameterList}, 
to make it simple to switch between different types of analysis. 
They also all inherit from the same base classes (response-only model evaluators) so that the resulting analysis can 
in turn be driven by non-intrusive analysis routines.

\todo{Perego: To be continued/rewritten}

\subsection{ROL}
Rapid Optimization Library (ROL) \cite{rol,ROL2022ICCOPT} is the
Trilinos library for numerical optimization. ROL brings an extensive
collection of state-of-the-art optimization algorithms to virtually
any application. Its programming interface supports any computational
hardware, including heterogeneous many-core systems with digital and
analog accelerators.
ROL has been used with great success for optimal control, optimal design,
inverse problems, image processing and mesh optimization, in application
areas including geophysics, climate science \cite{Perego2022}, structural
dynamics, fluid dynamics \cite{Antil2023}, electromagnetics, quantum
computing, hypersonics and geospatial imaging \cite{Kouri2014}.

The key design, performance and algorithmic features of ROL can be
summarized as follows:

\begin{itemize}
\item
\emph{Vector abstractions and matrix-free interface for universal applicability}.
Similar to other Trilinos packages that implement abstract numerical
algorithms, ROL's design is centered around an abstract linear algebra
interface, through the ROL::Vector class, which enables the use of any
ROL~algorithm with any data type, such as the C++ std::vector, MPI-parallel
data structures (e.g., Epetra, Tpetra, PETSc, HYPRE vectors), GPU-supporting data
structures (e.g., Kokkos, ArrayFire), etc.  The abstract linear algebra
interface further enables the definition of custom vector dot products,
which are critical for performance in large-scale applications.
Finally, all ROL algorithms are matrix-free and rely on user-defined
applications of linear and nonlinear operators.  ROL's vector class design is
inspired by the Hilbert Class Library \cite{hcl}, the Rice Vector Library \cite{rvl}
and the RTOp framework \cite{rtop}, while ROL's operator interfaces for
simulation-based optimization, known as the SimOpt middleware,
are related to \cite{Heinkenschloss1999}.
\item
\emph{Modern algorithms for unconstrained and constrained optimization}.
ROL features a large collection of well-established algorithms for smooth
optimization as well as several novel algorithms developed by the ROL team.
ROL categorizes optimization problems into 4 basic types, based on the presence of
certain constraints: Type U (unconstrained), Type B (bound constraints),
Type E (equality constraints) and Type G (general constraints). Additionally,
each problem type can include linear constraints, which ROL can reduce or eliminate
algorithmically. The novel algorithms include augmented Lagrangian methods for
infinite-dimensional problems with general constraints \cite{ALESQP} and
matrix-free trust-region methods for convex-constrained
optimization \cite{Kouri2022}.
\item
\emph{Easy-to-use methods for stochastic and risk-aware optimization}.
\item
\emph{Fast and robust algorithms for nonsmooth optimization}.
\item
\emph{Trust-region methods for inexact and adaptive computations}.
ROL controls inexact function evaluations and exploits adaptive computations
and multi-fidelity models through its trust-region methods. Several of these methods
have been pioneered by the ROL team.  Their implementations typically require some
knowledge of error in the computations, i.e., an \emph{error estimate}.
For a comprehensive overview of problem formulations, algorithms with error control,
and their ROL implementations, see \cite{Kouri2018}.
\item
\emph{PDE-OPT application development kit for PDE-constrained optimization}.
To build and optimize models based on partial differential equations (PDEs),
ROL offers an application development kit, called PDE-OPT. PDE-OPT is based on a
modular design with three layers: local finite element computations,
which encode the physical equations governing the optimization problem;
global computations, which utilize Tpetra data structures;
and an interface to ROL, enabling efficient optimization.
To solve PDE-constrained optimization problems, the users need only implement the
local finite element layer of PDE-OPT, i.e., the evaluations of weak forms on
a mesh element; PDE-OPT automates all remaining steps.
\end{itemize}



\subsection{Sacado}

Sacado \cite{SacadoURL,phipps2012efficient,phipps2008large} provides forward and reverse-mode operator overloading-based automatic differentiation (AD) tools within Trilinos.
%The package provides both forward and reverse-mode AD data types. 
Sacado forward AD tools have been integrated into Kokkos and have demonstrated good performance on GPU architectures~\cite{phipps2022automatic}.
Sacado, along with its Kokkos integration, provides high-performance derivative capabilities to numerous Office of Science and NNSA extreme scale applications, including Albany for solid mechanics and land ice modeling~\cite{Salinger2016,MPASAlbany2018}, 
Charon for semiconductor device modeling~\cite{CharonUsersManual2020} and multiphase chemically reacting flows~\cite{Musson2009}, Drekar for computational fluid dynamics (CFD)~\cite{Sondak2021,Shadid2016}, magnetohydrodynamics~\cite{Shadid2016mhd} and 
plasma physics~\cite{Crockatt2022,Miller2019}, Xyce for electronic circuit simulation~\cite{xyceTrilinos,xycePCE}, and SPARC for hypersonic fluid flows~\cite{SparcValidation}. 

\subsection{Stokhos}

Stokhos~\cite{phipps2015stokhos,Phipps2016,phipps2014exploring} provides implementations of two intrusive uncertainty quantification strategies: 
the intrusive stochastic Galerkin uncertainty quantification method~\cite{ghanem1990polynomial,ghanem2003stochastic} and the embedded ensemble propagation~\cite{phipps2017embedded}.

For the first one, Stokhos provides methods for computing intrusive stochastic Galerkin projections such as Polynomial Chaos and Generalized Polynomial Chaos, 
interfaces for forming the resulting nonlinear systems, and linear solver methods for solving block stochastic Galerkin linear systems.
The implementation targets GPU performances using Kokkos and by commuting the layout of the Galerkin operator to be outer-spatial and inner-stochastic~\cite{phipps2014exploring}.
The stochastic Galerkin implementation of Stokhos has been used in~\cite{constantine2014efficient} to efficiently propagate uncertainty in multiphysics systems by reducing the full system with a nonlinear elimination method.

The embedded ensemble propagation consists in propagating a subset of samples gathered into a so-called ensemble through the forward simulation at once.
It builds on~\cite{pawlowski2012automating} for automating embedded analysis capabilities; Stokhos defines an ensemble type, a SIMD data type, that is able to store
the values of the input, output, and state variables for every sample of an ensemble. This type can then be used in the Tpetra solver stack as a template argument for the scalar type.
This approach allows to save computation time in four ways: the sample-independent data and computation can be reused for every sample of an ensemble, the memory access pattern is improved,
the operations on the ensemble type can be vectorized efficiently, and the message passing costs are reduced by sending fewer but larger messages.
However, the approach requires solvers and BLAS functions to be aware of the extra dimension associated to the ensemble; for example, a GMRES for ensemble types~\cite{liegeois2020gmres} needs to monitor 
the convergence of the individual sample in order to decide when to stop based on the union of the information.


\section{Trilinos Framework}
\label{sec:framework}
\input{framework}

\section{Trilinos Community}
\label{sec:community}
\input{community}


\section{Discussion}


\newpage

\textcolor{red}{NOTE: Everything below here is just there for template from the journal, so folks know the style. This will be deleted in the end.}


\subsection{Template Parameters}

In addition to specifying the {\itshape template style} to be used in
formatting your work, there are a number of {\itshape template parameters}
which modify some part of the applied template style. A complete list
of these parameters can be found in the {\itshape \LaTeX\ User's Guide.}

Frequently-used parameters, or combinations of parameters, include:
\begin{itemize}
\item {\texttt{anonymous,review}}: Suitable for a ``double-blind''
  conference submission. Anonymizes the work and includes line
  numbers. Use with the \texttt{\acmSubmissionID} command to print the
  submission's unique ID on each page of the work.
\item{\texttt{authorversion}}: Produces a version of the work suitable
  for posting by the author.
\item{\texttt{screen}}: Produces colored hyperlinks.
\end{itemize}

This document uses the following string as the first command in the
source file:
\begin{verbatim}
\documentclass[acmsmall]{acmart}
\end{verbatim}

\section{Modifications}

Modifying the template --- including but not limited to: adjusting
margins, typeface sizes, line spacing, paragraph and list definitions,
and the use of the \verb|\vspace| command to manually adjust the
vertical spacing between elements of your work --- is not allowed.

{\bfseries Your document will be returned to you for revision if
  modifications are discovered.}

\section{Typefaces}

The ``\verb|acmart|'' document class requires the use of the
``Libertine'' typeface family. Your \TeX\ installation should include
this set of packages. Please do not substitute other typefaces. The
``\verb|lmodern|'' and ``\verb|ltimes|'' packages should not be used,
as they will override the built-in typeface families.

\section{Title Information}

The title of your work should use capital letters appropriately -
\url{https://capitalizemytitle.com/} has useful rules for
capitalization. Use the {\verb|title|} command to define the title of
your work. If your work has a subtitle, define it with the
{\verb|subtitle|} command.  Do not insert line breaks in your title.

If your title is lengthy, you must define a short version to be used
in the page headers, to prevent overlapping text. The \verb|title|
command has a ``short title'' parameter:
\begin{verbatim}
  \title[short title]{full title}
\end{verbatim}

\section{Authors and Affiliations}

Each author must be defined separately for accurate metadata
identification.  As an exception, multiple authors may share one
affiliation. Authors' names should not be abbreviated; use full first
names wherever possible. Include authors' e-mail addresses whenever
possible.

Grouping authors' names or e-mail addresses, or providing an ``e-mail
alias,'' as shown below, is not acceptable:
\begin{verbatim}
  \author{Brooke Aster, David Mehldau}
  \email{dave,judy,steve@university.edu}
  \email{firstname.lastname@phillips.org}
\end{verbatim}

The \verb|authornote| and \verb|authornotemark| commands allow a note
to apply to multiple authors --- for example, if the first two authors
of an article contributed equally to the work.

If your author list is lengthy, you must define a shortened version of
the list of authors to be used in the page headers, to prevent
overlapping text. The following command should be placed just after
the last \verb|\author{}| definition:
\begin{verbatim}
  \renewcommand{\shortauthors}{McCartney, et al.}
\end{verbatim}
Omitting this command will force the use of a concatenated list of all
of the authors' names, which may result in overlapping text in the
page headers.

The article template's documentation, available at
\url{https://www.acm.org/publications/proceedings-template}, has a
complete explanation of these commands and tips for their effective
use.

Note that authors' addresses are mandatory for journal articles.

\section{Rights Information}

Authors of any work published by ACM will need to complete a rights
form. Depending on the kind of work, and the rights management choice
made by the author, this may be copyright transfer, permission,
license, or an OA (open access) agreement.

Regardless of the rights management choice, the author will receive a
copy of the completed rights form once it has been submitted. This
form contains \LaTeX\ commands that must be copied into the source
document. When the document source is compiled, these commands and
their parameters add formatted text to several areas of the final
document:
\begin{itemize}
\item the ``ACM Reference Format'' text on the first page.
\item the ``rights management'' text on the first page.
\item the conference information in the page header(s).
\end{itemize}

Rights information is unique to the work; if you are preparing several
works for an event, make sure to use the correct set of commands with
each of the works.

The ACM Reference Format text is required for all articles over one
page in length, and is optional for one-page articles (abstracts).

\section{CCS Concepts and User-Defined Keywords}

Two elements of the ``acmart'' document class provide powerful
taxonomic tools for you to help readers find your work in an online
search.

The ACM Computing Classification System ---
\url{https://www.acm.org/publications/class-2012} --- is a set of
classifiers and concepts that describe the computing
discipline. Authors can select entries from this classification
system, via \url{https://dl.acm.org/ccs/ccs.cfm}, and generate the
commands to be included in the \LaTeX\ source.

User-defined keywords are a comma-separated list of words and phrases
of the authors' choosing, providing a more flexible way of describing
the research being presented.

CCS concepts and user-defined keywords are required for for all
articles over two pages in length, and are optional for one- and
two-page articles (or abstracts).

\section{Sectioning Commands}

Your work should use standard \LaTeX\ sectioning commands:
\verb|section|, \verb|subsection|, \verb|subsubsection|, and
\verb|paragraph|. They should be numbered; do not remove the numbering
from the commands.

Simulating a sectioning command by setting the first word or words of
a paragraph in boldface or italicized text is {\bfseries not allowed.}

\section{Tables}

The ``\verb|acmart|'' document class includes the ``\verb|booktabs|''
package --- \url{https://ctan.org/pkg/booktabs} --- for preparing
high-quality tables.

Table captions are placed {\itshape above} the table.

Because tables cannot be split across pages, the best placement for
them is typically the top of the page nearest their initial cite.  To
ensure this proper ``floating'' placement of tables, use the
environment \textbf{table} to enclose the table's contents and the
table caption.  The contents of the table itself must go in the
\textbf{tabular} environment, to be aligned properly in rows and
columns, with the desired horizontal and vertical rules.  Again,
detailed instructions on \textbf{tabular} material are found in the
\textit{\LaTeX\ User's Guide}.

Immediately following this sentence is the point at which
Table~\ref{tab:freq} is included in the input file; compare the
placement of the table here with the table in the printed output of
this document.

\begin{table}
  \caption{Frequency of Special Characters}
  \label{tab:freq}
  \begin{tabular}{ccl}
    \toprule
    Non-English or Math&Frequency&Comments\\
    \midrule
    \O & 1 in 1,000& For Swedish names\\
    $\pi$ & 1 in 5& Common in math\\
    \$ & 4 in 5 & Used in business\\
    $\Psi^2_1$ & 1 in 40,000& Unexplained usage\\
  \bottomrule
\end{tabular}
\end{table}

To set a wider table, which takes up the whole width of the page's
live area, use the environment \textbf{table*} to enclose the table's
contents and the table caption.  As with a single-column table, this
wide table will ``float'' to a location deemed more
desirable. Immediately following this sentence is the point at which
Table~\ref{tab:commands} is included in the input file; again, it is
instructive to compare the placement of the table here with the table
in the printed output of this document.

\begin{table*}
  \caption{Some Typical Commands}
  \label{tab:commands}
  \begin{tabular}{ccl}
    \toprule
    Command &A Number & Comments\\
    \midrule
    \texttt{{\char'134}author} & 100& Author \\
    \texttt{{\char'134}table}& 300 & For tables\\
    \texttt{{\char'134}table*}& 400& For wider tables\\
    \bottomrule
  \end{tabular}
\end{table*}

Always use midrule to separate table header rows from data rows, and
use it only for this purpose. This enables assistive technologies to
recognise table headers and support their users in navigating tables
more easily.

\section{Math Equations}
You may want to display math equations in three distinct styles:
inline, numbered or non-numbered display.  Each of the three are
discussed in the next sections.

\subsection{Inline (In-text) Equations}
A formula that appears in the running text is called an inline or
in-text formula.  It is produced by the \textbf{math} environment,
which can be invoked with the usual
\texttt{{\char'134}begin\,\ldots{\char'134}end} construction or with
the short form \texttt{\$\,\ldots\$}. You can use any of the symbols
and structures, from $\alpha$ to $\omega$, available in
\LaTeX~\cite{Lamport:LaTeX}; this section will simply show a few
examples of in-text equations in context. Notice how this equation:
\begin{math}
  \lim_{n\rightarrow \infty}x=0
\end{math},
set here in in-line math style, looks slightly different when
set in display style.  (See next section).

\subsection{Display Equations}
A numbered display equation---one set off by vertical space from the
text and centered horizontally---is produced by the \textbf{equation}
environment. An unnumbered display equation is produced by the
\textbf{displaymath} environment.

Again, in either environment, you can use any of the symbols and
structures available in \LaTeX\@; this section will just give a couple
of examples of display equations in context.  First, consider the
equation, shown as an inline equation above:
\begin{equation}
  \lim_{n\rightarrow \infty}x=0
\end{equation}
Notice how it is formatted somewhat differently in
the \textbf{displaymath}
environment.  Now, we'll enter an unnumbered equation:
\begin{displaymath}
  \sum_{i=0}^{\infty} x + 1
\end{displaymath}
and follow it with another numbered equation:
\begin{equation}
  \sum_{i=0}^{\infty}x_i=\int_{0}^{\pi+2} f
\end{equation}
just to demonstrate \LaTeX's able handling of numbering.

\section{Figures}

The ``\verb|figure|'' environment should be used for figures. One or
more images can be placed within a figure. If your figure contains
third-party material, you must clearly identify it as such, as shown
in the example below.
%\begin{figure}[h]
%  \centering
%  \includegraphics[width=\linewidth]{sample-franklin}
%  \caption{1907 Franklin Model D roadster. Photograph by Harris \&
%    Ewing, Inc. [Public domain], via Wikimedia
%    Commons. (\url{https://goo.gl/VLCRBB}).}
%  \Description{A woman and a girl in white dresses sit in an open car.}
%\end{figure}

Your figures should contain a caption which describes the figure to
the reader.

Figure captions are placed {\itshape below} the figure.

Every figure should also have a figure description unless it is purely
decorative. These descriptions convey what’s in the image to someone
who cannot see it. They are also used by search engine crawlers for
indexing images, and when images cannot be loaded.

A figure description must be unformatted plain text less than 2000
characters long (including spaces).  {\bfseries Figure descriptions
  should not repeat the figure caption – their purpose is to capture
  important information that is not already provided in the caption or
  the main text of the paper.} For figures that convey important and
complex new information, a short text description may not be
adequate. More complex alternative descriptions can be placed in an
appendix and referenced in a short figure description. For example,
provide a data table capturing the information in a bar chart, or a
structured list representing a graph.  For additional information
regarding how best to write figure descriptions and why doing this is
so important, please see
\url{https://www.acm.org/publications/taps/describing-figures/}.

\subsection{The ``Teaser Figure''}

A ``teaser figure'' is an image, or set of images in one figure, that
are placed after all author and affiliation information, and before
the body of the article, spanning the page. If you wish to have such a
figure in your article, place the command immediately before the
\verb|\maketitle| command:
\begin{verbatim}
  \begin{teaserfigure}
    \includegraphics[width=\textwidth]{sampleteaser}
    \caption{figure caption}
    \Description{figure description}
  \end{teaserfigure}
\end{verbatim}

\section{Citations and Bibliographies}

The use of \BibTeX\ for the preparation and formatting of one's
references is strongly recommended. Authors' names should be complete
--- use full first names (``Donald E. Knuth'') not initials
(``D. E. Knuth'') --- and the salient identifying features of a
reference should be included: title, year, volume, number, pages,
article DOI, etc.

The bibliography is included in your source document with these two
commands, placed just before the \verb|\end{document}| command:
\begin{verbatim}
  \bibliographystyle{ACM-Reference-Format}
  \bibliography{bibfile}
\end{verbatim}
where ``\verb|bibfile|'' is the name, without the ``\verb|.bib|''
suffix, of the \BibTeX\ file.

Citations and references are numbered by default. A small number of
ACM publications have citations and references formatted in the
``author year'' style; for these exceptions, please include this
command in the {\bfseries preamble} (before the command
``\verb|\begin{document}|'') of your \LaTeX\ source:
\begin{verbatim}
  \citestyle{acmauthoryear}
\end{verbatim}


  Some examples.  A paginated journal article \cite{Abril07}, an
  enumerated journal article \cite{Cohen07}, a reference to an entire
  issue \cite{JCohen96}, a monograph (whole book) \cite{Kosiur01}, a
  monograph/whole book in a series (see 2a in spec. document)
  \cite{Harel79}, a divisible-book such as an anthology or compilation
  \cite{Editor00} followed by the same example, however we only output
  the series if the volume number is given \cite{Editor00a} (so
  Editor00a's series should NOT be present since it has no vol. no.),
  a chapter in a divisible book \cite{Spector90}, a chapter in a
  divisible book in a series \cite{Douglass98}, a multi-volume work as
  book \cite{Knuth97}, a couple of articles in a proceedings (of a
  conference, symposium, workshop for example) (paginated proceedings
  article) \cite{Andler79, Hagerup1993}, a proceedings article with
  all possible elements \cite{Smith10}, an example of an enumerated
  proceedings article \cite{VanGundy07}, an informally published work
  \cite{Harel78}, a couple of preprints \cite{Bornmann2019,
    AnzarootPBM14}, a doctoral dissertation \cite{Clarkson85}, a
  master's thesis: \cite{anisi03}, an online document / world wide web
  resource \cite{Thornburg01, Ablamowicz07, Poker06}, a video game
  (Case 1) \cite{Obama08} and (Case 2) \cite{Novak03} and \cite{Lee05}
  and (Case 3) a patent \cite{JoeScientist001}, work accepted for
  publication \cite{rous08}, 'YYYYb'-test for prolific author
  \cite{SaeediMEJ10} and \cite{SaeediJETC10}. Other cites might
  contain 'duplicate' DOI and URLs (some SIAM articles)
  \cite{Kirschmer:2010:AEI:1958016.1958018}. Boris / Barbara Beeton:
  multi-volume works as books \cite{MR781536} and \cite{MR781537}. A
  couple of citations with DOIs:
  \cite{2004:ITE:1009386.1010128,Kirschmer:2010:AEI:1958016.1958018}. Online
  citations: \cite{TUGInstmem, Thornburg01, CTANacmart}.
  Artifacts: \cite{R} and \cite{UMassCitations}.

\section{Acknowledgments}

Identification of funding sources and other support, and thanks to
individuals and groups that assisted in the research and the
preparation of the work should be included in an acknowledgment
section, which is placed just before the reference section in your
document.

This section has a special environment:
\begin{verbatim}
  \begin{acks}
  ...
  \end{acks}
\end{verbatim}
so that the information contained therein can be more easily collected
during the article metadata extraction phase, and to ensure
consistency in the spelling of the section heading.

Authors should not prepare this section as a numbered or unnumbered {\verb|\section|}; please use the ``{\verb|acks|}'' environment.

\section{Appendices}

If your work needs an appendix, add it before the
``\verb|\end{document}|'' command at the conclusion of your source
document.

Start the appendix with the ``\verb|appendix|'' command:
\begin{verbatim}
  \appendix
\end{verbatim}
and note that in the appendix, sections are lettered, not
numbered. This document has two appendices, demonstrating the section
and subsection identification method.

\section{Multi-language papers}

Papers may be written in languages other than English or include
titles, subtitles, keywords and abstracts in different languages (as a
rule, a paper in a language other than English should include an
English title and an English abstract).  Use \verb|language=...| for
every language used in the paper.  The last language indicated is the
main language of the paper.  For example, a French paper with
additional titles and abstracts in English and German may start with
the following command
\begin{verbatim}
\documentclass[sigconf, language=english, language=german,
               language=french]{acmart}
\end{verbatim}

The title, subtitle, keywords and abstract will be typeset in the main
language of the paper.  The commands \verb|\translatedXXX|, \verb|XXX|
begin title, subtitle and keywords, can be used to set these elements
in the other languages.  The environment \verb|translatedabstract| is
used to set the translation of the abstract.  These commands and
environment have a mandatory first argument: the language of the
second argument.  See \verb|sample-sigconf-i13n.tex| file for examples
of their usage.

\section{SIGCHI Extended Abstracts}

The ``\verb|sigchi-a|'' template style (available only in \LaTeX\ and
not in Word) produces a landscape-orientation formatted article, with
a wide left margin. Three environments are available for use with the
``\verb|sigchi-a|'' template style, and produce formatted output in
the margin:
\begin{description}
\item[\texttt{sidebar}:]  Place formatted text in the margin.
\item[\texttt{marginfigure}:] Place a figure in the margin.
\item[\texttt{margintable}:] Place a table in the margin.
\end{description}

%%
%% The acknowledgments section is defined using the "acks" environment
%% (and NOT an unnumbered section). This ensures the proper
%% identification of the section in the article metadata, and the
%% consistent spelling of the heading.
\begin{acks}
To Robert, for the bagels and explaining CMYK and color spaces.
\end{acks}

%%
%% The next two lines define the bibliography style to be used, and
%% the bibliography file.
\bibliographystyle{ACM-Reference-Format}
\bibliography{bibliography}


%%
%% If your work has an appendix, this is the place to put it.
\appendix

\section{Research Methods}

\subsection{Part One}

Lorem ipsum dolor sit amet, consectetur adipiscing elit. Morbi
malesuada, quam in pulvinar varius, metus nunc fermentum urna, id
sollicitudin purus odio sit amet enim. Aliquam ullamcorper eu ipsum
vel mollis. Curabitur quis dictum nisl. Phasellus vel semper risus, et
lacinia dolor. Integer ultricies commodo sem nec semper.

\subsection{Part Two}

Etiam commodo feugiat nisl pulvinar pellentesque. Etiam auctor sodales
ligula, non varius nibh pulvinar semper. Suspendisse nec lectus non
ipsum convallis congue hendrerit vitae sapien. Donec at laoreet
eros. Vivamus non purus placerat, scelerisque diam eu, cursus
ante. Etiam aliquam tortor auctor efficitur mattis.

\section{Online Resources}

Nam id fermentum dui. Suspendisse sagittis tortor a nulla mollis, in
pulvinar ex pretium. Sed interdum orci quis metus euismod, et sagittis
enim maximus. Vestibulum gravida massa ut felis suscipit
congue. Quisque mattis elit a risus ultrices commodo venenatis eget
dui. Etiam sagittis eleifend elementum.

Nam interdum magna at lectus dignissim, ac dignissim lorem
rhoncus. Maecenas eu arcu ac neque placerat aliquam. Nunc pulvinar
massa et mattis lacinia.

\end{document}
\endinput
%%
%% End of file `sample-acmsmall.tex'.
