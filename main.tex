%%
%% This is file `sample-acmsmall.tex',
%% generated with the docstrip utility.
%%
%% The original source files were:
%%
%% samples.dtx  (with options: `acmsmall')
%%
%% IMPORTANT NOTICE:
%%
%% For the copyright see the source file.
%%
%% Any modified versions of this file must be renamed
%% with new filenames distinct from sample-acmsmall.tex.
%%
%% For distribution of the original source see the terms
%% for copying and modification in the file samples.dtx.
%%
%% This generated file may be distributed as long as the
%% original source files, as listed above, are part of the
%% same distribution. (The sources need not necessarily be
%% in the same archive or directory.)
%%
%%
%% Commands for TeXCount
%TC:macro \cite [option:text,text]
%TC:macro \citep [option:text,text]
%TC:macro \citet [option:text,text]
%TC:envir table 0 1
%TC:envir table* 0 1
%TC:envir tabular [ignore] word
%TC:envir displaymath 0 word
%TC:envir math 0 word
%TC:envir comment 0 0
%%
%%
%% The first command in your LaTeX source must be the \documentclass
%% command.
%%
%% For submission and review of your manuscript please change the
%% command to \documentclass[manuscript, screen, review]{acmart}.
%%
%% When submitting camera ready or to TAPS, please change the command
%% to \documentclass[sigconf]{acmart} or whichever template is required
%% for your publication.
%%
%%
\documentclass[acmsmall]{acmart}

%%
%% \BibTeX command to typeset BibTeX logo in the docs
\AtBeginDocument{%
  \providecommand\BibTeX{{%
    Bib\TeX}}}

%% Rights management information.  This information is sent to you
%% when you complete the rights form.  These commands have SAMPLE
%% values in them; it is your responsibility as an author to replace
%% the commands and values with those provided to you when you
%% complete the rights form.
\setcopyright{acmcopyright}
\copyrightyear{2025}
\acmYear{2025}
\acmDOI{XXXXXXX.XXXXXXX}

\usepackage{color}
\usepackage{todonotes}
\usepackage{siunitx} % \num
\sisetup{group-separator = {,}} % , sep for numbers
\let\oldtodo\todo
\renewcommand{\todo}[1]{
	\oldtodo[inline]{#1}
}

%\newcommand{\todo}[1]{\textcolor{orange}{TODO: #1}}
\DeclareMathOperator*{\argmin}{\arg\!\min}

\newcommand{\code}[1]{\texttt{#1}}

\usepackage{listings}
\lstset{language=C}


%%
%% These commands are for a JOURNAL article.
\acmJournal{JACM}
\acmVolume{37}
\acmNumber{4}
\acmArticle{111}
\acmMonth{8}

%%
%% Submission ID.
%% Use this when submitting an article to a sponsored event. You'll
%% receive a unique submission ID from the organizers
%% of the event, and this ID should be used as the parameter to this command.
%%\acmSubmissionID{123-A56-BU3}

%%
%% For managing citations, it is recommended to use bibliography
%% files in BibTeX format.
%%
%% You can then either use BibTeX with the ACM-Reference-Format style,
%% or BibLaTeX with the acmnumeric or acmauthoryear sytles, that include
%% support for advanced citation of software artefact from the
%% biblatex-software package, also separately available on CTAN.
%%
%% Look at the sample-*-biblatex.tex files for templates showcasing
%% the biblatex styles.
%%

%%
%% The majority of ACM publications use numbered citations and
%% references.  The command \citestyle{authoryear} switches to the
%% "author year" style.
%%

\usepackage{amsthm}
\theoremstyle{definition}
%\newtheorem{remark}{Remark}[chapter]
\newtheoremstyle{remarkstyle} % https://latex.org/forum/viewtopic.php?t=18631
  {}{}{}{}{\bfseries}{.}{.5em}{{\thmname{#1 }}{\thmnumber{#2}}{\thmnote{ (#3)}}}
\theoremstyle{remarkstyle}
\newtheorem{remark}{Remark}[section]


%%
%% end of the preamble, start of the body of the document source.
\begin{document}

%%
%% The "title" command has an optional parameter,
%% allowing the author to define a "short title" to be used in page headers.
\title{Trilinos: Enabling Scientific Computing across Diverse Hardware Architectures at Scale}

%%
%% The "author" command and its associated commands are used to define
%% the authors and their affiliations.
%% Of note is the shared affiliation of the first two authors, and the
%% "authornote" and "authornotemark" commands
%% used to denote shared contribution to the research.

\author{Matthias Mayr}
\authornote{These authors have contributed equally to this manuscript and, thus, share the first authorship.}
\authornote{Universit\"at der Bundeswehr M\"unchen, Werner-Heisenberg-Weg 39, 85577 Neubiberg, Germany}
%\affiliation{%
%  \institution{University of the Bundeswehr Munich}
%  \streetaddress{Werner-Heisenberg-Weg 39}
%  \city{85577 Neubiberg}
%  \country{Germany}}
\email{matthias.mayr@unibw.de}
\orcid{0000-0002-2780-1233}

\author{Alexander Heinlein}
\authornotemark[1]
\authornote{Corresponding author}
\authornote{Delft University of Technology, Mekelweg 4, 2628CD Delft, Netherlands}
%\affiliation{%
%  \institution{Delft University of Technology}
%  \streetaddress{Mekelweg 4}
%  \postcode{2628CD}
%  \city{Delft}
%  \country{Netherlands}
%}
\email{a.heinlein@tudelft.nl}
\orcid{0000-0003-1578-8104}

\author{Christian Glusa}\authornote{Sandia National Laboratories, 1515 Eubank Blvd SE, Albuquerque, NM 87123, United States}
\email{caglusa@sandia.gov}

\author{Sivasankaran Rajamanickam}\authornotemark[5]
\email{srajama@sandia.gov}
\orcid{1234-5678-9012}

\author{Maarten Arnst}
\email{maarten.arnst@uliege.be}
\authornote{University of Liege, Quartier Polytech 1, allée de la Découverte 9, 4000 Liege, Belgium}
\author{Roscoe A. Bartlett}\authornotemark[5]
\email{rabartl@sandia.gov}
\orcid{0000-0002-3831-8060}
\author{Luc Berger-Vergiat}\authornotemark[5]
\email{lberge@sandia.gov}
\orcid{0000-0001-5550-3527}
\author{Graham Harper}\authornotemark[5]
\email{gbharpe@sandia.gov}
\author{Michael Heroux}\authornote{ParaTools, Inc., 18125 Kreigle Lake, Road, Avon, MN 56310, United States}
\email{mheroux@paratools.com}
\orcid{0000-0002-5893-0273}
\author{Jonathan Hu}\authornotemark[5]
\email{jhu@sandia.gov}
\author{Drew P. Kouri}\authornotemark[5]
\email{dpkouri@sandia.gov}
\author{Paul Kuberry}\authornotemark[5]
\email{pakuber@sandia.gov}
\orcid{0000-0002-2426-4591}
\author{Kim Liegeois}\authornotemark[5]
\email{knliege@sandia.gov}
\orcid{0000-0002-1182-4078}
\author{Curtis C. Ober}\authornotemark[5]
\email{ccober@sandia.gov}
\author{Roger Pawlowski}\authornotemark[5]
\email{rppawlo@sandia.gov}
\author{Carl Pearson}\authornotemark[5]
\email{cwpears@sandia.gov}
\orcid{0000-0001-6481-970X}
\author{Mauro Perego}\authornotemark[5]
\email{mperego@sandia.gov}
\author{Eric Phipps}\authornotemark[5]
\email{etphipp@sandia.gov}
\author{Denis Ridzal}\authornotemark[5]
\email{dridzal@sandia.gov}
\author{Nathan V. Roberts}\authornotemark[5]
\email{nvrober@sandia.gov}
\orcid{0000-0003-1536-0749}
\author{Christopher M. Siefert}\authornotemark[5]
\email{csiefer@sandia.gov}
\author{Heidi K. Thornquist}\authornotemark[5]
\email{hkthorn@sandia.gov}
\author{Romin Tomasetti}\authornotemark[6]
\email{romin.tomasetti@uliege.be}
\author{Christian R. Trott}\authornotemark[5]
\email{crtrott@sandia.gov}
\author{Raymond S. Tuminaro}\authornotemark[5]
\email{rstumin@sandia.gov}
\author{James M. Willenbring}\authornotemark[5]
\email{jmwille@sandia.gov}
\orcid{0000-0002-0418-9264}
\author{Michael M. Wolf}\authornotemark[5]
\email{mmwolf@sandia.gov}
\author{Ichitaro Yamazaki}\authornotemark[5]
\email{iyamaza@sandia.gov}

%\affiliation{%
%  \institution{Sandia National Laboratories}
%  \streetaddress{1515, Eubank Blvd SE}
%  \city{Albuquerque}
%  \state{New Mexico}
%  \country{USA}
%  \postcode{87123}
%}

%\affiliation{%
%	\institution{University of Liege}
%	\streetaddress{Quartier Polytech 1, allée de la Découverte 9}
%	\postcode{4000}
%	\city{Liege}
%	\country{Belgium}
%}

%%
%% By default, the full list of authors will be used in the page
%% headers. Often, this list is too long, and will overlap
%% other information printed in the page headers. This command allows
%% the author to define a more concise list
%% of authors' names for this purpose.
\renewcommand{\shortauthors}{Mayr et al.}

%%
%% The abstract is a short summary of the work to be presented in the
%% article.
\begin{abstract}
Trilinos is a community-developed, open-source software framework that facilitates building large-scale, complex, multiscale, multiphysics simulation codes to tackle engineering and scientific problems.
Since the Trilinos framework has undergone substantial changes to support new applications and new hardware architectures,
this document is an update to ``An Overview of the Trilinos project'' by Heroux et al. (ACM Transactions on Mathematical Software, 31(3):397–423, 2005).
It describes the design of Trilinos, introduces its new organization in product areas,
and highlights established and new features available in Trilinos.
Particular focus is put on the modernized software stack based on the Kokkos ecosystem to deliver performance portability across heterogeneous hardware architectures.
This paper also contains notes on the Trilinos community and contribution model to help onboard interested users and contributors.
\end{abstract}

%%
%% The code below is generated by the tool at http://dl.acm.org/ccs.cfm.
%% Please copy and paste the code instead of the example below.
%%
%\begin{CCSXML}
%<ccs2012>
% <concept>
%  <concept_id>10010520.10010553.10010562</concept_id>
%  <concept_desc>Computer systems organization~Embedded systems</concept_desc>
%  <concept_significance>500</concept_significance>
% </concept>
% <concept>
%  <concept_id>10010520.10010575.10010755</concept_id>
%  <concept_desc>Computer systems organization~Redundancy</concept_desc>
%  <concept_significance>300</concept_significance>
% </concept>
% <concept>
%  <concept_id>10010520.10010553.10010554</concept_id>
%  <concept_desc>Computer systems organization~Robotics</concept_desc>
%  <concept_significance>100</concept_significance>
% </concept>
% <concept>
%  <concept_id>10003033.10003083.10003095</concept_id>
%  <concept_desc>Networks~Network reliability</concept_desc>
%  <concept_significance>100</concept_significance>
% </concept>
%</ccs2012>
%\end{CCSXML}

%\ccsdesc[500]{Computer systems organization~Embedded systems}
%\ccsdesc[300]{Computer systems organization~Redundancy}
%\ccsdesc{Computer systems organization~Robotics}
%\ccsdesc[100]{Networks~Network reliability}

%%
%% Keywords. The author(s) should pick words that accurately describe
%% the work being presented. Separate the keywords with commas.
\keywords{Scientific Software Frameworks; High Performance Computing}

\received{\today}
%\received[revised]{12 March 2009}
%\received[accepted]{5 June 2009}

%%
%% This command processes the author and affiliation and title
%% information and builds the first part of the formatted document.
\maketitle

\section{Introduction}

%\todo{Add introduction here. Add a subsection on Trilinos organization as a set of five products.}

Trilinos is a community-developed, open source software framework that facilitates building large-scale, complex, multiscale, multiphysics engineering and scientific problems. While Trilinos can run on small workstations to large supercomputers, the typical use of Trilinos is on the leadership class systems with new or emerging hardware architectures.

% History
Trilinos was originally conceived as framework of three packages for distributed memory systems. The original Trilinos publication~\cite{Heroux2005a} describes the motivation and the philosophy behind Trilinos and the capabilities that existed in Trilinos at that time. Trilinos today is similar to the Trilinos that was envisioned two decades ago in some aspects. However, Trilinos today is also very different in several other aspects. These changes were necessitated by the changes in programming models, application needs, hardware architectures, and algorithms. Trilinos has grown from a library of three packages to a library with more than fifty packages with functionality and features supporting a wide range of applications.

% Purpose
This article is an attempt to capture a snapshot of where Trilinos is today as opposed to eighteen years ago when the original Trilinos article was written. We will focus on the major developments within Trilinos in the last decade, new features and functionality that has been added to enable scientific and engineering applications. This article will be an overview of the features and we refer to the extensive reference list for the details of these features. We are also cognizant of the fact that as a software that is actively developed this article could become outdated even before its publication. Hence, we will focus on the high level features and project that we expect to remain stable for several years.

%Product and package structure
The functionalities in Trilinos are organized in two levels. The first one is \textit{package}. A package in Trilinos has a well-defined set of unique capabilities that is important for a scientific or an engineering application. Packages also have a set of expectations such as having a responsible point of contact or a package lead, software engineering expectations such as documentation, continuous integration testing, clearly defined dependencies, using the Trilinos infrastructure for building and installation etc. Recently, we have aggregated the fifty or more packages into five \textit{product areas} for organizational ease. The five product areas are data services, discretizations, linear solvers, embedded nonlinear analysis and tools, and framework. These product areas are collection of packages that share a common objective (e.g., solving a linear system), a sub-community within Trilinos, and in some cases common interfaces. We briefly describe these areas here.

\paragraph{Data Services} The data services product area covers all aspects of creating, distributing or mapping data to processing elements (cores, threads, nodes), load balancing, and redistributing data. Data services also includes Trilinos abstractions for data such as the Petra object model, and its concrete implementation called Tpetra. On a modern accelerator-based compute node the abstractions provided by the Kokkos library becomes critical for Tpetra. Section \ref{sec:data_services} describes these features in detail.
 
\paragraph{Discretization} \todo{Mauro} ....... Section \ref{sec:discretization} describes these features in detail.


\paragraph{Linear Solvers} The wide variety of applications that use Trilinos need a diverse set of linear solvers. Trilinos has support for both iterative and direct linear solvers. There are a number of preconditioner options from multithreaded or performance portable node-level preconditioners to scalable multilevel domain decomposition or multigrid preconditioners. The preconditioners and solvers use the data abstractions from the data services product area. Section \ref{sec:lin_solve} describes these features in detail.

\paragraph{Embedded Nonlinear Analysis Tools} The nonlinear analysis product area provides high level algorithms for computational simulation and design. Capabilities include solvers for nonlinear equations, time integration, parameter continuation, bifurcation tracking, optimization and uncertainty quantification. This capability area also provides lower level utility packages to evaluate quantities of interest required by the analysis algorithms. Capabilities include automatic differentiation technology to evaluate derivatives and embedded ensemble propagation for uncertainty quantification. Section \ref{sec:nonlin_solve} describes these features in detail.

\paragraph{Framework} \todo{Jim/Curt} ....... Section \ref{sec:framework} describes these features in detail.

%Article organization
This article describes Trilinos' product areas and their packages with a focus towards providing an overview of recent developments. We also briefly touch upon the Trilinos community (Section \ref{sec:community}) and software engineering issues with respect to Trilinos.



\section{Trilinos Core}
\label{sec:data_services}
\todo{Describe performance portability}

\subsection{Kokkos Kernels}\label{subsec:kk}
Kokkos Kernels~\cite{rajamanickam2021kokkoskernels} is part of the Kokkos ecosystem
~\cite{trott2021kokkos} and provides node local implementations of mathematical kernels
widely used across packages in Trilinos. As a member of the Kokkos
ecosystem, Kokkos Kernels is tightly integrated on Kokkos features and aims at
delivering performance portable algorithms across major CPU and GPU based HPC systems.
Due to its node local nature, Kokkos Kernels does not rely on MPI or other communication
libraries unlike numerous other packages in Trilinos.

The implementation of Kokkos Kernels algorithms leverage the hierarchical parallelism
exposed by the Kokkos library~\cite{kim2017designing} and increasingly provides coverage
for stream callable kernels. To ensure flexibility for the distributed libraries that
might call its algorithms, Kokkos Kernels provides thread safe and asynchronous
implementations for most of its kernels. Kokkos Kernels also serves as a major point of
integration for vendor optimized libraries such as cuBLAS, cuSPARSE, rocBLAS, rocSPARSE,
MKL, ARMpl and others.

The capabilities that Kokkos Kernels provides can be divided in four major categories:
1. BLAS algorithms, 2. sparse linear algebra and preconditioners, 3. graph algorithms, and
4. batched dense and sparse linear algebra~\cite{liegeois2023performance}. The main
points of integration of Kokkos Kernels in Trilinos are Tpetra for the dense and sparse
linear algebra capabilities, Ifpack2 for the preconditioners and batched algorithms,
the multigrid package MueLu that relies on these features both directly and indirectly
as well as on some specialized algorithms such as graph
coloring/coarsening~\cite{kelley2022parallel} and fused Jacobi-SpGEMM kernels.

Similarly to the Kokkos library, Kokkos Kernels is developed in its own GitHub
repository\footnote{https://github.com/kokkos/kokkos-kernels} outside of the Trilinos
GitHub repository. Every version of the library is integrated and tested in Trilinos
as part of the Kokkos ecosystem release process. Additional information on Kokkos
Kernels capabilities can be found here~\cite{deveci2018multithreaded,wolf2017fast}.


\subsection{Tpetra}\label{subsec:tpetra}
Tpetra \cite{hoemmen2015tpetra} provides the distributed-memory
infrastructure for sparse linear algebra computations.  It implements
distributed-memory linear algebra objects, such as sparse graphs,
sparse matrices and dense vectors, using Kokkos for local data
storage.  Distributed-memory sparse linear algebra operations, such as
a sparse matrix-vector product, are implemented through on-node calls
to Kokkos Kernels and inter-node MPI communication.   Tpetra features
include:
\begin{itemize}
\item \textit{Maps} --- distributions of objects over MPI ranks.
\item \textit{(Multi)Vectors} --- storage of dense vectors or collections of
vectors (multivectors) and associated BLAS-1 like kernels (e.g., dot
products, norms, scaling, vector addition, pointwise vector
multiplication) as well as tall skinny QR (TSQR)  factorization for multivectors.
\item \textit{Import/export} --- moving vector, graph and matrix data
between different distributions (maps).  This is key for performing
halo/boundary exchanges, as well as other kernels such as
sparse matrix-matrix multiplication.
\item \textit{Sparse graphs} --- in compressed sparse row (CSR)
format.  Graphs also include import/export objects for use in
halo/boundary exchanges associate with the graph.
\item \textit{Sparse matrices} --- in compressed sparse row (CSR) and
block compressed sparse row (BSR) formats.  Associated kernels include
sparse matrix vector product (SPMV), sparse matrix-matrix
multiplication and triple-product, sparse matrix-matrix addition, sparse matrix transpose, diagonal extraction,
Frobenius norm calculation, and row/column scaling.
\end{itemize}


\section{Linear Solvers and Preconditioners}
\label{sec:lin_solve}
% !TEX root = main.tex

%\todo{Describe new linear solver packages, ShyLU, FROSch, new packages similar to old packages, new capabilities in these packages (two level DD), GPU supported features}
%
%\todo{@Siva/Alexander: do this!}

Trilinos offers many linear solver capabilities: dense and sparse direct solvers, iterative solvers, shared-memory preconditioners local to a compute node, and scalable distributed memory domain decomposition and multigrid methods. Furthermore, interfaces to several third-party direct solvers are provided. All native solver capabilities are built on top of Kokkos and are GPU capable to varying degrees; any exceptions are noted in the detailed descriptions below.




\subsection{Iterative Linear Solvers and Krylov Methods: Belos}

Belos~\cite{Bavier2012a} provides next-generation iterative linear solvers and a powerful developer framework for solving linear systems and least-squares problems. This framework provides several abstractions that facilitate code reuse and extensibility.

First, Belos decouples the algorithms from the implementation of the underlying linear algebra object using traits mechanisms.  While concrete linear algebra adapters are provided for Tpetra and Thyra, users can also implement their own interfaces to leverage any existing matrix and vector representations. Second, there are abstractions to orthogonalization to ease the integration of application or architecture-specific orthogonalization methods. Implementations of iterated classical Gram-Schmidt, DGKS-corrected (Daniel, Gragg, Kaufman, and Stewart-corrected) classical Gram-Schmidt, and iterated modified Gram-Schmidt algorithms are provided.  Finally, powerful solver managers encapsulate the strategy for solving a linear system or least-squares problem using abstract interfaces to iteration kernels.  As a result, the algorithms developed using Belos abstractions can be relatively agnostic to data layout in memory or distributed over processors and parallel matrix/vector operations.

Belos supersedes the AztecOO package~\cite{Heroux2004a} providing solver managers for single-vector iterative solvers, but also extensions of these methods to block iterative methods for solving linear systems with multiple right-hand sides.  Single-vector solver managers include conjugate gradient (CG), minimum residual (MINRES), generalized minimal residual (GMRES), stabilized biconjugate gradient (BiCGStab), and transpose-free QMR (TFQMR) as well as ``seed'' solvers (hybrid GMRES, PCPG), subspace recycling solvers (GCRO-DR and RCG), and least-squares solvers (LSQR).  There are versions of CG, GMRES, and GCRO-DR that construct a block Krylov subspace to solve for one or more right-hand sides.  There are also ``pseudo-block'' versions of CG, GMRES, and TFQMR that apply the single-vector iteration simultaneously on multiple right-hand sides, where matrix and preconditioner applications are aggregated to achieve better performance. Any single-vector iteration can be used to solve multiple right-hand sides as well but, if a block or pseudo-block version is not used, the solver will sequentially solve the right-hand sides.

In recent years, application or architecture-specific variants of the classic iterative methods have been integrated into Belos. Some of these variations are so minor that they are enabled through input parameters on the classic method. For instance, flexible GMRES~\cite{Saad1993a} is an option for the block GMRES solver, and pipelined CG~\cite{GHYSELS2014224} is an option for both the block CG and pseudo-block CG solver.  Additionally, there are stand-alone solvers for a pseudo-block stochastic CG method~\cite{Parker2012SamplingGD} and a fixed-point iteration method.


\subsection{One-Level Domain Decomposition and Basic Iterative Methods: Ifpack2}

Trilinos provides domain decomposition approaches in two different
packages: Ifpack2 and ShyLU (specifically the ShyLU\_DD and FROSch
subpackages). Ifpack2 implements overlapping additive Schwarz
approaches with several options for the local subdomain solves. The
local subdomain solvers may be either CPU-only versions of incomplete
factorization preconditioners implemented in Ifpack2 itself, such as
ILU(k) and ILUt (thresholded ILU), or architecture-portable algorithms
for incomplete factorizations and triangular solvers implemented in
Kokkos Kernels. It is also possible to use direct solvers as subdomain
solvers, and ShyLU also supports shared-memory inexact
incomplete factorization preconditioners.
%
Such one-level preconditioners can be used as smoothers within multigrid
methods, for solving ``simpler'' problems where the setup cost of more robust and scalable multilevel methods is prohibitive in comparison to the reduction in the number of iterations, or when the underlying problem is simply not amenable to multilevel methods.

Ifpack2 also supplies classic iterative methods based on matrix-splitting techniques, such as Jacobi iteration, Gauss-Seidel, and an MPI-oriented hybrid of Jacobi and Gauss-Seidel (e.g., Jacobi between ranks and Gauss-Seidel on them). Ifpack2 also provides preconditioners
based on Chebyshev iterations. The aforementioned preconditioners are available both in point and block
forms and can operate on CRS and BSR matrices. In the block case,
line relaxation is also supported, while in the point case, techniques
like Vanka relaxation \cite{Vanka1986} are possible.  Auxiliary-space
smoothing for $H(curl)$ and $H(div)$ discretizations in the style of
Hiptmair \cite{Hiptmair1997} are also supported.
Local kernels are either implemented in Ifpack2 itself
but can also be called from Kokkos Kernels for performance\hyp{}portable shared-memory algorithmic variants of Gauss-Seidel and Jacobi iterations.

\subsection{Multilevel Domain Decomposition Methods: FROSch}
\label{ssec:frosch}

FROSch (Fast and Robust Overlapping Schwarz) is a framework for the construction of multilevel Schwarz domain decomposition preconditioners. Besides parallel scalability, FROSch emphasizes applicability and robustness across a wide range of challenging problems while supporting an algebraic construction. Specifically, most preconditioners can be built using only the fully assembled system matrix, though some variants can take advantage of additional geometric inputs. The algebraic construction relies on two key algorithmic components: first, the creation of an overlapping domain decomposition at the initial level based on the sparsity pattern of the system matrix, similar to Ifpack2; and second, the integration of extension-based coarse spaces, as in the classical two-level Generalized Dryja--Smith--Widlund (GDSW) preconditioner~\cite{dohrmann_domain_2008} and its variants. While the initial version of FROSch~\cite{heinlein_parallel_2016} was based on the outdated Epetra linear algebra framework, the current implementation~\cite{heinlein_frosch_2020} leverages Xpetra. Originally designed as a lightweight wrapper around Epetra and Tpetra, the Xpetra package facilitated over the years compatibility with both the Epetra and Tpetra stacks. With the deprecation of Epetra, it now exclusively supports the Tpetra stack. Algorithmic variants of Schwarz methods implemented in FROSch include:
\begin{itemize}
	\item \emph{Extension-based coarse spaces based on a partition of unity on the interface}, such as classical GDSW, reduced dimension GDSW (RGDSW) coarse spaces, and multiscale finite element method (MsFEM) coarse spaces; cf.~\cite{heinlein_parallel_2016,heinlein_improving_2018};
	\item \emph{Monolithic Schwarz preconditioners} for block systems; cf.~\cite{heinlein_monolithic_2019}.
	\item \emph{Multilevel Schwarz preconditioners}, which are obtained from two-level Schwarz preconditioners by recursively applying Schwarz preconditioners as an inexact solver for the coarse problems; cf.~\cite{heinlein_parallel_2022}.
\end{itemize}

FROSch has been applied to various challenging application problems, including: scalar elliptic and elasticity problems~\cite{heinlein_parallel_2016}, possibly with heterogeneities~\cite{alves2024computationalstudyalgebraiccoarse}; computational fluid dynamics problems~\cite{heinlein_monolithic_2019,heinlein_comparison_2025}; time-harmonic Maxwell's and fluid-structure interaction problems~\cite{heinlein2024couplingdealiifroschsustainable}; pharmaco-mechanical interactions in arterial walls~\cite{balzani_computational_nodate}; and coupled multiphysics problems for land ice simulations~\cite{heinlein_frosch_2022}. The latter three have been solved using monolithic preconditioning techniques. To extend robustness for heterogeneous model problems, an implementation of spectral coarse spaces~\cite{heinlein_adaptive_2019} is currently under development. FROSch preconditioners have scaled to more than \num{200000} cores on the Theta Cray XC40 supercomputer at the Argonne Leadership Computing Facility (ALCF); cf.~\cite{heinlein_parallel_2022}.

In its current implementation, FROSch assumes a one-to-one correspondence of subdomains and MPI ranks. Using an interface to the other solver packages in Trilinos, a variety of inexact solvers can be employed for the solution of the local subdomain problems. An extension to multiple subdomains per MPI rank is currently being implemented. Using Kokkos and Kokkos Kernels, FROSch has recently also been ported to GPUs~\cite{yamazaki_experimental_2023} with performance gains for the triangular solve or inexact solves with ILU on GPUs.

A demo/tutorial for FROSch can be found at the GitHub repository~\cite{frosch_demo}.

\subsection{Multigrid Methods: MueLu}

MueLu is a flexible and scalable high-performance multigrid solver library.
It provides a variety of multigrid algorithms for problems ranging from Poisson-like operators, over elasticity, convection-diffusion, Navier-Stokes, and Maxwell's equations,
all the way to multigrid methods for coupled multiphysics systems.
Besides its strong focus on aggregation-based algebraic multigrid (AMG) methods,
MueLu comes with specialized capabilities for (semi-)structured grids to perform semi-coarsening along grid lines
while forming the coarse operator via a Galerkin product (in contrast to classical geometric multigrid methods).
MueLu is extensible and allows for the research and development of new multigrid preconditioning methods.
Its weak and strong scalability up to \num{131000} cores of a Cray XC40 and one million cores of a Blue Gene/Q system, even for vector-valued partial differential equations (PDEs) on unstructured meshes, have been shown in~\cite{Lin2017a,Thomas2019a}.

MueLu provides several approaches to constructing and solving the multilevel problem:

\begin{itemize}
\item \emph{Algebraic smoothed aggregation approach}~\cite{Vanek1996a}:
The matrix graph is colored to create aggregates (groups) of nodes.
These aggregates define a tentative projection operator.
A final projection operator is created by applying a smoother to the tentative operator.
There are a variety of deterministic and non-deterministic coloring algorithms implemented directly
in MueLu and in Kokkos Kernels.  For a full description, see~\cite{BergerVergiat2023a}.

\item \emph{Algebraic multigrid for Maxwell's equations}:
  MueLu implements specialized solvers for the solution of curl-curl problems~\cite{BochevHuEtAl2008_AlgebraicMultigridApproachBased}.
  Scaling results on Haswell, KNL, ARM and NVIDIA V100 GPUs at full machine scale can be found in~\cite{BettencourtBrownEtAl2021_EmpirePic}.

\item \emph{Multigrid for multiphysics problems}:
MueLu implements a tool box to compile multi-level block preconditioners for block matrices arising from coupled multiphysics problems.
Applications range from Navier-Stokes equations
over surface-coupling (as in fluid\hyp{}structure interaction or contact mechanics~\cite{Wiesner2021a})
to volume-coupled problems (e.g., in magneto-hydro dynamics~\cite{Ohm2022a}).

\item \emph{Semi-coarsening algebraic multigrid approach}~\cite{Tuminaro2016a}:
Specialized aggregation procedures for three-dimensional meshes generated by extrusion of a two-dimensional unstructured grid
allow to first coarsen in the direction of extrusion to reduce the system to a two-dimensional representation and then perform classical aggregation-based AMG
for the remaining coarsenings.

\item \emph{AMG for (semi-)structured grids}:
MueLu has a structured aggregation capability in which the user may specify the coarsening rate used to compute interpolation operators.
The coarse operators are then formed via a Galerkin product to avoid remeshing on the coarse levels.
This work has been extended to semi-structured grids to leverage structured-grid computational performance also for globally unstructured grids~\cite{Mayr2022a}.

\item \emph{Geometric multigrid / matrix-free capabilities}:
MueLu treats the objects in its hierarchy as operators instead of matrices whenever possible.
This enables one to use MueLu in a matrix-free fashion provided the user can supply their own grid transfers and coarse operators.
% through factories such as \texttt{MatrixFreeTentativePFactory},
% which only requires aggregates in order to perform a grid transfer using a tentative prolongator.
% Other grid transfers such as structured grid transfers discussed above may also be performed in a matrix-free fashion,
% and simple smoothers such as Jacobi and Chebyshev may also be performed on matrix-free operators.

\end{itemize}

MueLu has a required dependency on Kokkos,
and the main code paths in MueLu (e.g., smoothed aggregation, semi-coarsening, $p$-coarsening, structured aggregation)
utilize Kokkos and have been shown to scale on device \cite{BettencourtBrownEtAl2021_EmpirePic}.
Some of the earlier algorithms such as energy minimization~\cite{Sala2008a} and the variable degree-of-freedom Laplacian have not been adapted to fully utilize Kokkos yet,
however, they are part of ongoing efforts to use Kokkos throughout MueLu completely and to merge serial and device-capable algorithms.

There are a variety of resources to learn more about MueLu.
An overview is given on the MueLu website\footnote{\url{https://trilinos.github.io/muelu.html}}.
The MueLu User's Guide~\cite{BergerVergiat2023a} summarizes installation instructions and a reference to most of MueLu's configuration parameters.
The MueLu Tutorial~\cite{Mayr2023b} introduces beginners and experts to various topics in MueLu and shows how to solve or precondition different linear systems using MueLu.
Details on the compatibility of MueLu and its predecessor ML~\cite{Heroux2005a,Gee2006a} can be found in the MueLu User's Guide~\cite{BergerVergiat2023a}.

Besides its C++ API, MueLu offers a MATLAB interface, MueMex, to provide access to MueLu's aggregation and solver routines from MATLAB.
MueMex allows users to setup and solve arbitrarily many problems with either MueLu as a preconditioner, Belos as a solver and Epetra or Tpetra for data structures.

MueLu requires the packages Teuchos, Tpetra, Xpetra, Kokkos, Kokkos Kernels, Amesos2, Ifpack2, and Zoltan2. Additionally, it supports interfaces to abstraction layer packages such as Stratimikos and Thyra through the MueLu adapters library. These interfaces are required to use MueLu with the Teko block preconditioning package. For more details on using MueLu with Teko, see the MueLu examples directory in Trilinos.

\subsection{Direct Linear Solvers: ShyLU and Amesos2}

ShyLU~\cite{ShyLUCore2014} provides an implementation of three solvers: Basker, Tacho, and a distributed-memory solver based on the Schur complement method.

Basker~\cite{Basker2017} is a sparse direct solver based on LU factorization for the problems that have the block triangular form (BTF) typically seen in circuit simulation applications. Basker uses this structure to factor and solve the diagonal blocks in parallel. The larger diagonal blocks can themselves be factored in parallel by discovering the parallelism available using a nested-dissection reordering. Basker focuses on exploiting thread-parallelism on the multi-core CPU architectures. %However, we have still implemented the solver using Kokkos.
Amesos2 also has a templated implementation of a sequential KLU solver called KLU2, which also exploits the BTF structure.

Tacho is a sparse direct solver that exploits supernodal block structure commonly found in sparse direct factorizations of matrices from mechanics applications. Tacho exploits this supernodal structure for both factorization and triangular-solve phases. It is based on Kokkos. Originally, Tacho implemented a task-parallel Cholesky factorization of sparse symmetric positive definite (SPD) matrices~\cite{Tacho2018}. However, to improve its portability, it has been extended to compute the sparse factorization based on level-set scheduling. Moreover, its functionality has been extended to the computation of an LDLT factorization of symmetric indefinite matrices, as well as the computation of an LU factorization of general matrices with a symmetric sparsity pattern.

In addition to their stand-alone use, the aforementioned node-level solvers may be used as the local solvers for domain decomposition preconditioners (Ifpack2 or FROSch) or as the coarse solvers for multilevel preconditioners (MueLu or FROSch).

ShyLU also implements a distributed-memory linear solver based on the Schur complement method~\cite{ShyLUCore2014}. This hybrid approach combines direct and iterative solvers: each subdomain problem is solved in parallel using a direct solver, while the Schur complement is handled iteratively. The preconditioner for the Schur complement solver is computed using a probing approach or using a threshold-based dropping strategy. It has been developed to address the requirements of circuit simulation applications. The solver is also hybrid in terms of parallel computing, utilizing MPI+threads. However, it is not designed for GPU architectures, and therefore, it has not been implemented using Kokkos. % This solver has been shown to be useful for circuit simulation applications.

Amesos2~\cite{Bavier2012a} provides a uniform interface to direct linear solvers.
These include third-party direct solvers such as CHOLMOD, MUMPS, Pardiso\_MKL, SuperLU, SuperLU\_MT, SuperLU\_Dist, and STRUMPACK.
Amesos2 also provides an interface to the on-node sparse direct solvers implemented in ShyLU.



\subsection{Physics Block Operators and Preconditioners: Teko}
\label{sec:teko}

The Teko library~\cite{Cyr2016a} provides interfaces for operators and preconditioners that are constructed from large physics-based sub-blocks.
The sub-blocks are Thyra operators which are themselves often implemented using Tpetra matrices.
Teko provides generic block preconditioning strategies such as $2\times2$ block-LU factorizations as well as block versions of Jacobi- and Gauss--Seidel-type methods for any number of blocks,
as well as commonly used approximate inverse strategies for the Navier-Stokes equation
such as SIMPLEC, LSC, and PCD preconditioners~\cite{CyrShadidEtAl2012_StabilizationScalableBlockPreconditioning}.
The Block-LU class offers the possibility to implement user-specific inverse approximation strategies,
for example using a sparse approximate inverse to construct the Schur complement~\cite{Firmbach2024a}.
More complicated multilevel hierarchies of block solvers can be generated via \code{Teuchos::ParameterList} objects.
Block preconditioners for first-order formulations of Maxwell's equations and Darcy flow are implemented in the Panzer package.
Teko's solvers can be registered in the Stratimikos interface for usage entirely driven by \code{Teuchos::ParameterList} objects.

\subsection{Eigensolvers: Anasazi}
Anasazi~\cite{Baker2009a} is an extensible and interoperable framework for large-scale eigenvalue algorithms.
This framework provides a generic interface to a collection of algorithms that are built upon abstract interfaces
that facilitate code reuse and extensibility.  Similar to Belos, Anasazi decouples the algorithms from the
implementation of the underlying linear algebra objects using traits mechanisms.  Concrete linear algebra adapters
are provided for Tpetra and Thyra, while users can also implement their own interfaces to leverage any existing matrix and vector representations. Any libraries that understand Tpetra and Thyra matrices
and vectors, like Belos and Ifpack2, may also be used in conjunction with Anasazi.  The suite of eigensolvers provided
by Anasazi includes locally optimal block preconditioned conjugate gradient (LOBPCG), block Davidson, Riemannian Trust-Region
(RTR), and block Krylov-Schur methods.  Recently, a family of trace minimization (TraceMin) methods and a
generalized Davidson method have been added to the suite of eigensolvers in Anasazi.


\subsection{Unified Solver Interface: Stratimikos}
\label{sec:Stratimikos}
The Stratimikos package offers a cohesive interface to various linear solvers and preconditioners within Trilinos, including Amesos2, Belos, FROSch, Ifpack2, MueLu, and Teko.
It requires that the matrix, right-hand side, and the solution vectors all support the Thyra interface.
Thyra also provides wrappers for Tpetra linear algebra.
Users can specify solver and preconditioner parameters via a \code{Teuchos::ParameterList}, which can be easily populated from an XML file.
Moreover, the package allows for the easy addition of new solver and preconditioner factories via adapters, with examples available in the \texttt{packages/stratimikos/adapters} directory.


\section{Nonlinear Solvers and Analysis Tools}
\label{sec:nonlin_solve}

The Nonlinear Analysis product area provides the top level algorithms for a computational simulation or design study.
These include nonlinear solvers, time integration, bifurcation tracking, parameter continuation, optimization, and uncertainty quantification.
A common theme of this collection is the philosophy of ``analysis beyond simulation'', which aims to automate many computational tasks that are often performed by application code users by trial-and-error or repeated simulation.
Tasks that can be automated include performing parameter studies, sensitivity analysis, calibration, optimization, time step size control, and locating instabilities.
This capability area additionally includes utilities for the nonlinear analysis. These include automatic differentiation tools that can provide the derivatives critical to the analysis algorithms and the abstraction layers and interfaces for application callbacks.

\subsection{Thyra}
\todo{Bartlett}

\subsection{NOX}
NOX provides algortihms for solving large\-scale sets of nonlinear equations.
The library contains abstractions for solvers, directions and line searches that allow users to customize their code.
The stopping criteria has a set 
Methods include Newton-based algorithms such as inexact Newton, matrix-free Newton-Krylov, line-search methods, trust-region methods, tensor methods, and homotopy methods.
\todo{Pawlowski, finish this}

\subsection{LOCA}
LOCA~\cite{Salinger2005}, short for the Library of Continuation Algorithms, provides techniques for computing families of solutions to nonlinear equations as well as methods for investigating their stability when these nonlinear equations define equilibria of dynamical systems.  It builds on the NOX nonlinear solver package to track solutions to sets of nonlinear equations as a function of one or more parameters (continuation).  Given an interface to NOX defining the nonlinear equations, all users must additionally provide is an ability to set the parameter values that are being varied.  LOCA provides several continuation methods, including pseudo-arclength continuation which allows tracking solution curves around turning points/folds.  Furthermore, LOCA has hooks to call the Anasazi eigensolver package to estimate leading eigenvalues of the linearization at each point along the continuation curve for linear stability analysis, including various spectral transformations highlighting eigenvalues in different regimes (e.g., largest magnitude, largest real, ...).  Finally, LOCA implements equations augmenting the original nonlinear equations to locate and track bifurcation points where linear stability changes (e.g., turning point, pitchfork, and Hopf bifurcations) as a function of additional parameters.

\subsection{Tempus}

Tempus (Latin meaning time as in “tempus fugit” -> “time flies”)
is the Trilinos time-integration package for advanced transient
analysis.  It includes various time integrators and embedded
sensitivity analysis for next-generation code architectures.  Tempus
provides “out-of-the-box” time-integration capabilities, which
allows users to quickly and easily incorporate time-integration
capabilities to their applications and switch between various time
integrators depending on the simulation needs.  Additionally, Tempus
provides “build-your-own” capabilities, which allows applications
to incorporate various Tempus components to augment or replace
application transient capabilities. Other capabilities include
embedded error analysis, sensitivity analysis, transient optimization
with ROL.

Tempus provides a general infrastructure for the time evolution of
solutions through a variety of general integration schemes.  Tempus
provides time integrators for explicit and implicit methods and for
first- and second-order ODEs.  It can be used from small systems of
equations (e.g., single ODEs for the time evolution of plasticity
models, and multiple ODEs for coupled chemical reactions) to
large-scale transient simulations requiring exascale computing
(e.g., flow fields around reentry vehicles and magneto-hydrodynamics).

Tempus has several components that can be used in concert or
individually, depending on the needs of the application.
\begin{itemize}
  \item Integrators are the time-loop structure for time integration
  and provide several features, e.g., control the advancement of
  the solution, selection of the next timestep size and solution
  output.

  \item Time Steppers are individual methods that advance the
  solution from one step to the next.  A variety of time steppers
  are available:
  \begin{itemize}
    \item Classic one-step methods (e.g., Forward Euler and Trapezoidal
    Method)
    \item Explicit Runge-Kutta methods (e.g., RK Explicit 4 Stage)
    \item Diagonally Implicit Runge-Kutta (DIRK) Methods (e.g.,
    general tableau DIRK and many specific DIRK/SDIRK methods)
    \item Implicit-Explicit (IMEX) Runge-Kutta Methods (e.g., IMEX
    RK SSP2, IMEX RK SSP3, and general tableau IMEX RK methods)
    \item Multi-Step Methods (i.e., BDF2)
    \item Second-order ODE Methods (e.g., Leapfrog, Newmark methods
    and HHT-Alpha)
    \item Steppers with subSteppers (e.g., operator-split and
    subcycling methods)
  \end{itemize}

  \item Solution History is used to maintain the solution during
  time-step failure, solution restart/output, interpolation of
  solution between time steps, and to provide the solution for
  transient adjoint sensitivities.

  \item Timestep Control and Strategies provide methods to select
  the time-step size based on user input and/or temporal error
  control (e.g., bounding min/max time-step size, relative/absolute
  maximum error, and timestep adjustments for output and checkpointing)

\end{itemize}
Additionally, Tempus has several mechanisms which allow users to
insert application-specific algorithms into Tempus components (e.g.,
through observers and creating derived classes).

\subsection{Piro}

Piro~\cite{osti_1231283} is the top-level, unifying package of the embedded nonlinear analysis capability area. 
The main purpose of the package is to provide driver classes for the common uses of Trilinos nonlinear analysis tools. 
These drivers all can be constructed similarly, with a \lstinline{Thyra::ModelEvaluator} and a \lstinline{Teuchos::ParameterList}, 
to make it simple to switch between different types of analysis. 
They also all inherit from the same base classes (response-only model evaluators) so that the resulting analysis can 
in turn be driven by non-intrusive analysis routines.

\todo{Perego: To be continued/rewritten}

\subsection{ROL}
\todo{Ridzal}

\subsection{Sacado}

Sacado \cite{SacadoURL,phipps2012efficient,phipps2008large} provides forward and reverse-mode operator overloading-based automatic differentiation (AD) tools within Trilinos.
%The package provides both forward and reverse-mode AD data types. 
Sacado forward AD tools have been integrated into Kokkos and have demonstrated good performance on GPU architectures~\cite{phipps2022automatic}.
Sacado, along with its Kokkos integration, provides high-performance derivative capabilities to numerous Office of Science and NNSA extreme scale applications, including Albany for solid mechanics and land ice modeling~\cite{Salinger2016,MPASAlbany2018}, 
Charon for semiconductor device modeling~\cite{CharonUsersManual2020} and multiphase chemically reacting flows~\cite{Musson2009}, Drekar for computational fluid dynamics (CFD)~\cite{Sondak2021,Shadid2016}, magnetohydrodynamics~\cite{Shadid2016mhd} and 
plasma physics~\cite{Crockatt2022,Miller2019}, Xyce for electronic circuit simulation~\cite{xyceTrilinos,xycePCE}, and SPARC for hypersonic fluid flows~\cite{SparcValidation}. 

\subsection{Stokhos}

Stokhos~\cite{phipps2015stokhos,Phipps2016,phipps2014exploring} provides implementations of two intrusive uncertainty quantification strategies: 
the intrusive stochastic Galerkin uncertainty quantification method~\cite{ghanem1990polynomial,ghanem2003stochastic} and the embedded ensemble propagation~\cite{phipps2017embedded}.

For the first one, Stokhos provides methods for computing intrusive stochastic Galerkin projections such as Polynomial Chaos and Generalized Polynomial Chaos, 
interfaces for forming the resulting nonlinear systems, and linear solver methods for solving block stochastic Galerkin linear systems.
The implementation targets GPU performances using Kokkos and by commuting the layout of the Galerkin operator to be outer-spatial and inner-stochastic~\cite{phipps2014exploring}.
The stochastic Galerkin implementation of Stokhos has been used in~\cite{constantine2014efficient} to efficiently propagate uncertainty in multiphysics systems by reducing the full system with a nonlinear elimination method.

The embedded ensemble propagation consists in propagating a subset of samples gathered into a so-called ensemble through the forward simulation at once.
It builds on~\cite{pawlowski2012automating} for automating embedded analysis capabilities; Stokhos defines an ensemble type, a SIMD data type, that is able to store
the values of the input, output, and state variables for every sample of an ensemble. This type can then be used in the Tpetra solver stack as a template argument for the scalar type.
This approach allows to save computation time in four ways: the sample-independent data and computation can be reused for every sample of an ensemble, the memory access pattern is improved,
the operations on the ensemble type can be vectorized efficiently, and the message passing costs are reduced by sending fewer but larger messages.
However, the approach requires solvers and BLAS functions to be aware of the extra dimension associated to the ensemble; for example, a GMRES for ensemble types~\cite{liegeois2020gmres} needs to monitor 
the convergence of the individual sample in order to decide when to stop based on the union of the information.


\section{Discretizations}
\label{sec:discretization}
% !TEX root = main.tex

This section describes Trilinos packages that provide tools for spatial and temporal discretization of integro-differential equations. Most of Trilinos discretization efforts have been devoted to implement tools for mesh-based discretizations of partial differential equations (PDEs) with a focus on high-order finite elements. A notable exception is constituted by the research package Compadre, which provides tools for meshless approximation of linear operators that can be used for the discretization of differential equations and for data transfer.
Trilinos discretization packages have been adopted by many applications addressing a wide range of physics problems, including solid mechanics, earth system modeling, semiconductor devices and electro-magnetics. These applications have taken different approaches in adopting Trilinos mesh-based discretization tools. The less intrusive approach is the adoption of Intrepid2 tools to perform local finite element assembly. The application has to manage the global assembly possibly using the DoFManager provided by Panzer and FE Crs matrix and vector structures provided by Tpetra.  
A more intrusive approach is to additionally use Phalanx package for managing dependencies of field evaluations in conjunction with the Thyra Model Evaluator and Sacado algorithmic differentiations, and possibly Tempus for time integration. This approach is particularly useful when developing complex multiphysics problems because it allows easy re-use of computational kernels and automates the computation of Jacobian and sensitivities.
The most intrusive approach is to build the application around the Panzer package, which provides all of the above, plus the handling of linear and nonlinear solvers and integrated constrained optimization capabilities.
In the following we describe some of the Trilinos discretization packages. For brevity, we do not include the description of these packages: Sierra ToolKit (STK), Krino and Percept, which provide mesh and level-set tools, and Shards, which provides tools for mesh topology. We refer to Trilinos website (\url{https://trilinos.github.io}) for a brief description of these packages.



\subsection{Intrepid2}
Intrepid2 provides interoperable tools for compatible discretizations of PDEs; it is a performance-portable re-implementation and extension of the legacy Intrepid package \cite{bochev2012}. Intrepid2 mainly focuses on local assembly of continuous and discontinuous finite elements. It also provides limited capabilities for finite volume discretization.  Intrepid2 works on batches of elements (cells), and provides tools to efficiently compute discretized linear functionals (e.g., right-hand-side vectors) and differential operators (e.g., stiffness matrices) at the element level. Intrepid2 implements compatible finite element spaces of various polynomial orders for $H({\rm grad})$, $H({\rm curl})$, $H({\rm div})$ and $L^2$ function spaces on triangles, quadrilaterals, tetrahedrons, hexahedrons, wedges and pyramids. It provides both Lagrangian basis functions and hierarchical basis functions \cite{fuentes2015} and it implements performance optimizations (e.g., sum factorizations) exploiting the underlying structure of the problem (e.g., tensor-product elements or other symmetries).  The degrees of freedom of $H({\rm div})$ and $H({\rm curl})$ finite elements, as well as high-order $H({\rm grad})$ finite elements, depend on the global orientation of edges and faces and Intrepid2 provides orientation tools for matching the degrees of freedom on shared edges and faces. It also provides interpolation-based projection tools for projecting functions in $H({\rm grad})$, $H({\rm curl})$, $H({\rm div})$ and $L^2$ to the respective discrete spaces. Intrepid2 implements these capabilities through these classes:
\begin{itemize}
\item \emph{CellTools:} This class provides geometric operations on the reference and physical frame. Includes computation of tangents and normals to edges/faces in the physical frame, computation of Jacobian of the reference-to-physical frame maps, and other metric computations. 
\item \emph{CubatureFactory:} This class provides several quadrature rules (called \emph{cubatures} in Intrepid2) of various degrees of accuracy for approximating integrals over the elements and their boundaries.
\item \emph{Basis:} This is the base class for a variety of basis functions for compatible finite element spaces. Each class includes a \texttt{getValues()} method that computes the value of the basis functions or their derivatives (e.g., gradient for $H({\rm grad})$ functions, curl for $H({\rm curl})$ functions) at a set of input points. The implementation of \texttt{getValues()} can be very different depending on the basis. Specific optimizations are available for tensor-product elements.  Additionally, there is a \texttt{BasisFamily} class with a convenience method, \texttt{getBasis()}, which constructs a basis depending on a template argument specifying the type of basis (hierarchical or nodal, e.g.), the cell topology and function space on which it is defined, and its polynomial degree.
\item \emph{OrientationTools:} This class provides methods to orient the basis functions based on the global orientation of edges and faces, determined by the global numbering of the cell vertices. This is achieved by building a linear operator (a permutation for tensor-product elements) that encodes the orientation of a particular cell, and applying that operator to the reference basis functions.
\item \emph{ProjectionTools:} This class provides methods for interpolation-based projections of a given function into a compatible finite element space or between compatible finite element spaces \cite{demkowicz2007}.  The provided projections commute with the corresponding differential operators if the quadrature rules can exactly integrate the functions being projected. As an example, projecting an $H({\rm grad})$ function into the $H({\rm grad})$ finite-element space and then taking its gradient gives the same result as taking the gradient of the function first, and then projecting the gradient into the $H({\rm curl})$ finite-element space.
\item \emph{FunctionSpaceTools:} This class provides transformations of fields from reference to physical frame and back, computation of measures on edges, faces and cells, scalar/vector/tensor multiplications and contractions for computing integrals.
\item \emph{IntegrationTools:} This class provides integration methods that can take advantage of tensor product structures in basis values, providing mechanisms for performance-portable, \emph{sum-factorized} assembly across $H({\rm grad})$, $H({\rm curl})$, $H({\rm div})$ and $L^2$ function spaces.  In the future, we plan to provide similar interfaces to support matrix-free discretizations.
\end{itemize}
Intrepid2 makes use of Kokkos containers to enable memory layouts that are adapted to the computational platform. Intrepid2 also uses Kokkos for its core computational kernels, enabling threaded execution across a variety of architectures. The data types used by Intrepid2 are templated; it is therefore possible to propagate Sacado types through Intrepid2 to perform automatic differentiation. Current development of Intrepid2 focuses on providing efficient matrix-free discretizations to enhance efficiency on GPU architectures. 

\subsection{Phalanx}
Phalanx is a local field evaluation library designed for equation assembly in partial differential equation (PDE) applications. The goal of Phalanx is to decompose a complex problem into a number of simpler problems with managed dependencies to support rapid development and extensibility of PDE codes \cite{Notz2012,pawlowski2012automating,pawlowski2012automatingpart2}. The data structures use Kokkos \cite{trott2021kokkos} for performance portability. Through the use of template metaprogramming, Phalanx supports arbitrary user defined data types and evaluation types. This feature allows for simple integration of automatic differentiation tools via operator overloaded scalar types from the Sacado library \cite{phipps2022automatic}. From a simple equation definition, quantities such as Jacobians, Jacobian-vector products, Hessians and parameter sensitivities can be evaluated to machine precision. These quantities can be used by other Trilinos packages for operations including Newton-based nonlinear solves, gradient-based optimization, constraint enforcement and bifurcation analysis \cite{pawlowski2012automating,pawlowski2012automatingpart2}.

Phalanx uses a graph-based design to manage data dependencies. The runtime defined directed acyclic graph allows for rapid prototyping in a production environment where simple interfaces for analysts, flexible models/data structures and integration of non-trivial Third-party libraries are paramount. Phalanx is used in a number of large scale parallel codes including Albany \cite{Salinger2016}, Charon \cite{CharonUsersManual2020}, Drekar \cite{Crockatt2022,Miller2019,Shadid2016mhd} and EMPIRE \cite{BettencourtBrownEtAl2021_EmpirePic}.

In recent years, Phalanx has been extended to provide utilities for performance portability under automatic differentiation. For example, Phalanx provides tools for building and managing a Kokkos view-of-views on device without the use of unified virtual memory and provides utilities for running virtual functions on device. In the future, these utilities may be split out into a separate package.

\subsection{Tempus}
Tempus (Latin meaning time as in “tempus fugit” $\rightarrow$ “time flies”)
is the Trilinos time-integration package for advanced transient
analysis.  It includes various time integrators and embedded
sensitivity analysis for next-generation code architectures.  Tempus
provides “out-of-the-box” time-integration capabilities, which
allows users to quickly and easily incorporate time-integration
capabilities to their applications and switch between various time
integrators depending on the simulation needs.  Additionally, Tempus
provides “build-your-own” capabilities, which allows applications
to incorporate various Tempus components to augment or replace
application transient capabilities. Other capabilities include
embedded error analysis, sensitivity analysis, transient optimization
with ROL.

Tempus provides a general infrastructure for the time evolution of
solutions through a variety of general integration schemes.  Tempus
provides time integrators for explicit and implicit methods and for
first- and second-order ODEs.  It can be used from small systems of
equations (e.g., single ODEs for the time evolution of plasticity
models, and multiple ODEs for coupled chemical reactions) to
large-scale transient simulations requiring exascale computing
(e.g., flow fields around reentry vehicles and magneto-hydrodynamics).

Tempus has several components that can be used in concert or
individually, depending on the needs of the application.
\begin{itemize}
  \item Integrators are the time-loop structure for time integration
  and provide several features, e.g., control the advancement of
  the solution, selection of the next timestep size and solution
  output.

  \item Time Steppers are individual methods that advance the
  solution from one step to the next.  A variety of time steppers
  are available:
  \begin{itemize}
    \item Classic one-step methods (e.g., Forward Euler and Trapezoidal
    Method)
    \item Explicit Runge-Kutta methods (e.g., RK Explicit 4 Stage)
    \item Diagonally Implicit Runge-Kutta (DIRK) Methods (e.g.,
    general tableau DIRK and many specific DIRK/SDIRK methods)
    \item Implicit-Explicit (IMEX) Runge-Kutta Methods (e.g., IMEX
    RK SSP2, IMEX RK SSP3, and general tableau IMEX RK methods)
    \item Multi-Step Methods (i.e., BDF2)
    \item Second-order ODE Methods (e.g., Leapfrog, Newmark methods
    and HHT-Alpha)
    \item Steppers with subSteppers (e.g., operator-split and
    subcycling methods)
  \end{itemize}

  \item Solution History is used to maintain the solution during
  time-step failure, solution restart/output, interpolation of
  solution between time steps, and to provide the solution for
  transient adjoint sensitivities.

  \item Timestep Control and Strategies provide methods to select
  the time-step size based on user input and/or temporal error
  control (e.g., bounding min/max time-step size, relative/absolute
  maximum error, and timestep adjustments for output and checkpointing)

\end{itemize}
Additionally, Tempus has several mechanisms which allow users to
insert application-specific algorithms into Tempus components (e.g.,
through observers and creating derived classes).

\subsection{Panzer}
The Panzer package provides global tools for finite element analysis. The package is targeted for large-scale parallel implicit PDEs using continuous and discontinuous compatible finite elements as well as control volume finite elements. Panzer enables the solution of nonlinear problems, by interfacing with several Trilinos linear and nonlinear solvers. It uses the Intrepid2 package for arbitrary order finite element bases. It computes derivatives and sensitivities with the Sacado automatic differentiation (AD) package. Panzer supports both Epetra and Tpetra data structures and achieves performance portability through the Kokkos programming model.

Panzer is designed for multiphysics systems. The assembly engine allows for different equation sets in each element block of the mesh and allows for mixed bases for degrees of freedom (DOF) within each element block (e.g. mixed HDiv and HCurl system for electromagnetics). The assembly tools can create a single fully coupled Jacobian matrix with all off-block dependencies (as a single Tpetra::CrsMatrix) or it can create a blocked system, grouping sets of DOFs into separate explicit matrices. This allows for physics-based block preconditioning strategies using the Teko block preconditioner package in Trilinos \cite{Bonilla2023,Cyr2016a}. The assembly process relies on the Phalanx package for efficient assembly of the multiphysics systems. Panzer additionally can wrap the assembly in a Thyra::ModelEvaluator, providing direct interfaces to the linear and nonlinear analysis packages in Trilinos.

Panzer can provide low level utilities for application codes to build on, or can be used as a high level application framework. Important capabilities include the following:
\begin{itemize}
\item \emph{DOF Manager:} Panzer provides a stand-alone DOF manager class in the dof-mgr subpackage. Given a list of DOFs and their corresponding basis and element blocks, the DOF manager can provide the mapping from DOFs on the mesh entities to the entries in linear algebra objects such as a residual vector or Jacobian matrix. The DOF manager can return the objects required to build distributed Tpetra and Epetra maps and graphs for both uniquely owned global indices and ghosted indices used during assembly. It provides the local indexing used during assembly as well. The DOF manager contains a mesh abstraction called a connection manager that provides information about mesh connectivity, global numbering of mesh topological entities and element block groups. It is designed to support any underlying mesh database, allowing applications to use the DOF manager with any finite element application.
\item \emph{STK\_Interface:} Panzer contains a concrete implementation of the connection manager API for the STK mesh database package in Trilinos. The STK\_Interface object wraps a STK mesh database. It provides a simple interface for accessing global indices on the mesh and can be used to read/write associated solution data to the database. It additionally supports SEACAS use for writing mesh data to disk. This STK\_Interface also includes support for periodic boundary conditions.  This capability can match topological entities on periodic parallel distributed faces. Once matched, the DOFs are unified to enforce periodicity for the DOF manager. This capability is in the adapters-stk subpackage.
\item \emph{Linear Object Factory:} Panzer provides a linear object factory and linear object container designed to support parallel distributed assembly. It supports both Epetra and Tpetra objects. The returned containers hold the linear algebra objects for either a uniquely owned DOF map used for solving the linear system or a ghosted version of the linear objects that can be used for assembly. The containers support export and import operations between the unique and ghosted containers. These are used to simplify the assembly process and abstract the underlying linear objects (i.e. Epetra or Tpetra). The linear object factory can also create DOF gather and scatter functors for the corresponding (and possibly blocked) matrices. The gather and scatter operations are specialized on the assembly types, including residuals, Jacobians, Tangents and Hessians. Hessians are only supported for Epetra at the moment but will be expanded to Tpetra as needed. The capability is found in the disc-fe subpackage.
\item \emph{Worksets:} The Panzer assembly process provides workset containers to hold static data that can be reused in finite element computations. For example, when using a static mesh, these containers could hold the Jacobian, Jacobian inverse and all finite element basis values at the quadrature points. The workset concept breaks the local elements on an MPI process into smaller blocks of uniform computations that can be dispatched to a CPU or GPU. The workset helps to control total memory use for temporaries in the Phalanx directed acyclic graph. This capability is in the disc-fe subpackage.
\item \emph{Assembly Kernels:} Panzer provides a number of assembly kernels that are optimized for use with Sacado AD data types. When assembling on GPUs, the assembly kernels were found to be an order of magnitude faster if the AD evaluation was parallelized over the internal derivative dimension \cite{phipps2022automatic}. This required using kernels with a Kokkos::TeamPolicy for finite element assembly where the lowest level of parallelism is used internally by the Sacado AD type. When building Panzer for GPUs, the DFad hierarchic parallelism flag in Sacado should be enabled for kernel performance. The kernels can be found in the disc-fe subpackage. 
\item Panzer provides an example miniapp for implicit electromagnetics that is used for benchmarking linear solver performance and for acceptance testing of new high performance computing systems. It demonstrates an HCurl-HDiv formulation for the electric and magnetic fields. This is found in the MiniEM subpackage.
\end{itemize}

The Panzer package is intended to provide both low and high level tools for implicit finite elements discretizations. The high level tools aggregate many Trilinos discretization and solver packages. While the high level tools can be used as a rapid prototyping environment, the front end is fairly complex to set up as opposed to true rapid prototyping frameworks such as DealII \cite{dealII95}, FEniCSx \cite{BarattaEtal2023} and Firedrake \cite{FiredrakeUserManual}. Panzer is not user friendly in this regard, however the examples and miniapps are a good starting point and can be quickly adapted to other physics. A number of performance portable applications are using Panzer tools at different levels of adoption, these include Albany \cite{Salinger2016}, Charon \cite{CharonUsersManual2020}, Drekar \cite{Crockatt2022,Miller2019,Shadid2016mhd} and EMPIRE \cite{BettencourtBrownEtAl2021_EmpirePic}.

\subsection{Compadre}
The Compadre package provides tools for the approximation of linear operators (including point evaluation and derivatives) given the location of samples of a function over an unstructured cloud of points. The resulting stencils, when applied directly to samples of the function at these locations, provides an approximation of the linear operator acting on the function at the point(s) queried. This is useful for meshed and meshless data transfer applications. Samples of the function at the specified locations can also be viewed as unknowns, in which case the solution returned by Compadre can be used as a stencil for meshless discretization of PDEs. 

The package uses generalized moving least squares (GMLS) for approximating functionals.
Classic moving least squares consists of approximating continuous target functions evaluated over a cloud of points by solving weighted least square problems to locally (within a neighborhood of the evaluation point) approximate the function with polynomials. The generalized approach implemented in Compadre allows to possibly replace the evaluation of the target function at points with \emph{sampling} functionals of such function (e.g. integrals of the function over local domains), and, instead of approximating the target function, to approximate a \emph{target} operator (e.g. gradient, curl, divergence) of such function. Another generalization involves using basis functions other than polynomials. Detailed description of GMLS can be found in \cite{mirzaei2012generalized,wendland2004scattered}. Future plans include the implementation of other meshless methods like radial basis functions.

%Consider $\phi$ of function class $\mathbf{V}$. Consider a collection of samples $\Lambda = \left\{\lambda_i(\phi)\right\}_{i=1}^{N}$ corresponding to a quasi-uniform\cite{wendland2004scattered} collection of data sites $\mathbf{X}_h = \left\{\mathbf{x}_i\right\} \subset \mathbb{R}^d$ characterized by fill distance $h$. To approximate a given linear target functional $\tau_{\tilde{x}}$ associated with a target site $\tilde{x}$, we seek a reconstruction $p \in \mathbf{V}_h$, where $\mathbf{V}_h \subset \mathbf{V}$ is a finite dimensional space chosen to provide good approximation properties, with basis $\mathbf{P}=\left\{P\right\}_{i=1}^{dim(V_h)}$. We perform this reconstruction in the following weighted $\ell_2$ sense:

%\begin{equation}
%\label{gmls}
%p = \underset{{q \in \mathbf{V}_h}}\argmin \sum_{i=1}^N \left( \lambda_i(\phi) -\lambda_i(q) \right)^2 \omega(\lambda_i,\tau_{\tilde{x}})
%\end{equation}
%where $\omega$ is a locally supported positive function. Compadre offers several choices for the weighting kernel $\omega = \Phi(|\tilde{x}-\mathbf{x}_i|)$, where $|\cdot|$ denotes the Euclidean norm.

%With the optimal reconstruction $p$ computed, the target functional is approximated via $\tau_{\tilde{x}} (\phi) \approx \tau^h_{\tilde{x}} (\phi) := \tau_{\tilde{x}} (p)$. As an unconstrained $\ell_2$-optimization problem, this process admits the explicit form


%\begin{equation}
%\label{discreteTarget}
%\tau^h_{\tilde{x}}(\phi) = \tau_{\tilde{x}}(\mathbf{P})^\intercal \left(\Lambda(\mathbf{P})^\intercal \mathbf{W} \Lambda(\mathbf{P})\right)^{-1} \Lambda(\mathbf{P})^\intercal \mathbf{W} \Lambda(\phi),
%\end{equation}
%where we denote:
%\begin{itemize}
%  \item $\tau_{\tilde{x}}(\mathbf{P}) \in \mathbb{R}^{dim(V_h)}$ is a vector with components consisting of the target functional applied to each basis function.
%  \item $\mathbf{W} \in \mathbb{R}^{N \times N}$ is a diagonal matrix with diagonal entries consisting of $\left\{\omega(\lambda_i,\tau_{\tilde{x}})\right\}_{i=1,...,N}$.
%  \item $\Lambda(\mathbf{P}) \in \mathbb{R}^{N \times dim(V_h)}$ is a rectangular matrix whose $(i,j)$ entry corresponds to the application of the $i^{th}$ sampling functional applied to the $j^{th}$ basis function.
%  \item $\Lambda(\phi) \in \mathbb{R}^N$ is a vector consisting of the $N$ samples of the function $\phi$.
%\end{itemize}
%We note that by taking the contraction of the tensors appearing in \eqref{discreteTarget} and exploiting the compact support of $\omega$, we may interpret the output of the GMLS process as a finite difference-like stencil of the form 
%\begin{equation}
%\tau^h_{\tilde{x}}(\phi) = \sum_{\mathbf{x}_i \in B^\epsilon(\tilde{x})} \alpha_i \lambda_i(\phi),
%\end{equation}
%where $B^\epsilon(\tilde{x})$ denotes the $\epsilon$-ball neighborhood of the target site $\tilde{x}$. Therefore, GMLS admits an interpretation as an automated process for generating generalized finite difference methods on unstructured point clouds. The computational cost of solving the GMLS problem amounts to inverting a small linear system which may be assembled using only information from neighbors within the support of $\omega$, and construction of such stencils across the entire domain is embarrassingly parallel. 

Compadre allows users to control the weighting kernel, degree of basis functions, the sampling functionals, and the target operator; this allows control over smoothness of the reconstruction, locality of the resulting stencil, order of accuracy of the reconstruction (assuming regularity of the function embedded in the point cloud data), choice of what the sampled data or degrees of freedom represent, and linear operator action, respectively.

Selecting point evaluations for the sampling functionals and the target operator provides a traditional moving least squares reconstruction. As an example of a more exotic choice, it is possible to use an average vector normal integral over edges as the sampling functionals and a cell average integral as the target operator, enabling recovery of functions embedded in a Raviart-Thomas type representation and transferring them to a basis consistent with a finite-volume scheme.

While Compadre supports full space reconstruction in 1-3D, there is also additional support for select sampling functionals and target operators on 1D smooth manifolds embedded in 2D or 2D smooth manifolds embedded in 3D. Reconstruction on a manifold is done through an on-the-fly Principal Component Analysis calculation to determine principal directions tangent to the manifold. The curvature of the manifold is calculated through a reconstruction of the tangent and normal components to the calculated tangent plane, and the final function reconstruction is performed in the local chart. Utilities in the package handle mappings between local computed charts and the ambient higher-dimensional space.

Compadre's stencil generation involves independent problems to be solved in parallel at the team level with loops over the thread and vector level within each problem. This hierarchical parallelism is achieved with performance portability by using the Kokkos programming model and leveraging the batched QR with pivoting algorithm implemented in Kokkos Kernels.


\section{Trilinos Framework}
\label{sec:framework}
\todo{Describe the modern framework, packages like pyTrilinos}
\todo{CG: PyTrilinos has currently no developer.}

\section{Trilinos Community}
\label{sec:community}
% !TEX root = main.tex

\subsection{Contributing to Trilinos}

Contributions to Trilinos can be offered through the standard GitHub pull request model.
Proposed code changes are required to pass a set of tests as well as review and approval prior to be merged.
Detailed instructions can be found in the contributing guidelines in the source code repository.

\subsection{Platforms for exchange among users or developers}

The community of Trilinos users and developers operates several forums for exchange and discussion.
Technical discussions about the source code and its development happen within the Trilinos GitHub repository\footnote{\url{https://github.com/trilinos/Trilinos}}.
Several mailing lists (see \url{https://trilinos.github.io/mail_lists.html}) distribute relevant information and updates on Trilinos.
The \texttt{\#trilinos} channel within the Kokkos slack workspace\footnote{See \url{https://kokkos.org} for details.} provides a quick and accessible forum to ask questions.

For in-person exchange, the \emph{Trilinos User-developer Group (TUG) Meeting} takes place at Sandia National Laboratories in Albuquerque every year.
At TUG, all Trilinos users and developers can come together to inform themselves about recent progress and advances,
discuss current challenges and upcoming topics relevant to the entire Trilinos community.

The \emph{European Trilinos User Group (EuroTUG) Meeting} series\footnote{\url{https://eurotug.github.io}}
offers a platform for Europe-based users and developers of the Trilinos project.
EuroTUG facilitates easy access to the Trilinos community and reduced travel burdens for Europe-based researchers and application engineers
who are interested in the Trilinos project.
It covers tutorial sessions to educate the community,
user presentations to demonstrate capabilities and features of various application codes using Trilinos,
and updates from developers to spread news and ongoing work to all interested parties.

\subsection{Embedding into other software initiatives}

Trilinos is a founding member of the High Performance Software Foundation\footnote{\url{https://hpsf.io}} (HPSF).
Established in 2024 under the Linux Foundation,
the HPSF aims to build, promote, and advance a portable software stack for HPC by fostering collaboration among industry, academia, and government entities.
As an initial technical project within the HPSF, Trilinos contributes its expertise in data structures for parallel computing, linear, nonlinear, and transient solvers,
as well as optimization and uncertainty quantification in support of HPFS’s mission to enhance the HPC software ecosystem as a whole.



\section{Concluding remarks}
\label{sec:conclusion}
% !TEX root = main.tex

The evolution of the Trilinos project demonstrates its critical role in scientific and high-performance computing.
This update underscores Trilinos' dedication to maintaining relevance amidst the rapid development of HPC software frameworks and hardware architectures, and the demands of increasingly complicated multiscale multiphysics simulation codes for tackling engineering and scientific problems.
By adopting the Kokkos ecosystem, expanding package functionalities, and embracing a modular structure,
Trilinos ensures both performance portability and adaptability to existing and emerging application domains.

The integration of data structures for distributed-memory paradigms, advanced linear and nonlinear solvers, discretization technology, and optimization tools highlights its versatility and impact across scientific and engineering disciplines.
Its community-driven development model fosters innovation while ensuring robust support for users and contributors alike.
Moving forward, the library's commitment to collaboration, scalability, and innovation positions it as a cornerstone of next-generation computational frameworks.

As the computational landscape evolves, Trilinos remains poised to address emerging challenges, leveraging its rich feature set and strong community foundation to drive progress in scientific discovery and technological innovation.

\section*{Acknowledgments}

Sandia National Laboratories is a multimission laboratory managed and operated by National Technology \& Engineering Solutions of Sandia, LLC, a wholly owned subsidiary of Honeywell International Inc., for the U.S. Department of Energy’s National Nuclear Security Administration under contract DE-NA0003525.

This paper describes objective technical results and analysis. Any subjective views or opinions that might be expressed in the paper do not necessarily represent the views of the U.S. Department of Energy or the United States Government.

This work was supported in part by the U.S. Department of Energy, Office of Science, Office of Advanced Scientific Computing Research, Scientific Discovery through Advanced Computing (SciDAC) Program through the FASTMath Institute.

This work was supported by the U.S.~Department of Energy, Office of Science, Office of Advanced Scientific Computing Research, Applied Mathematics program.

This work was supported by the Laboratory Directed Research and Development program at Sandia National Laboratories.

This work was supported by dtec.bw -- Digitalization and Technology Research Center of the Bundeswehr [project hpc.bw]. dtec.bw is funded by the European Union -- NextGenerationEU.

%%
%% The next two lines define the bibliography style to be used, and
%% the bibliography file.
\bibliographystyle{ACM-Reference-Format}
\bibliography{bibliography}

\end{document}
\endinput
%%
%% End of file `sample-acmsmall.tex'.
